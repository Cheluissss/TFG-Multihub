% ==========================================================
% CAPÍTULO 4: ESPECIFICACIÓN FUNCIONAL
% ==========================================================
\chapter{Análisis Funcional y Subsistemas}
\label{ch:funcionalidad}

\section{Modelo de Datos}

El núcleo del sistema se articula en torno a las siguientes entidades principales y sus relaciones:

\begin{itemize}
    \item \textbf{User:} Representa cada usuario del sistema. Campos: \texttt{id}, \texttt{email}, \texttt{role} (ADMIN/MANAGER/EMPLOYEE), \texttt{password\_hash}, \texttt{sede\_id} (referencia a sede asignada).
    
    \item \textbf{Sede:} Sucursal física de la organización. Campos: \texttt{id}, \texttt{nombre}, \texttt{ubicación}, \texttt{manager\_id} (FK User), \texttt{created\_at}.
    
    \item \textbf{Shift (Turno):} Período de trabajo asignado a un empleado. Campos: \texttt{id}, \texttt{employee\_id} (FK User), \texttt{sede\_id} (FK Sede), \texttt{start\_time}, \texttt{end\_time}, \texttt{status} (PENDING/CONFIRMED/CANCELLED), \texttt{date}.
    
    \item \textbf{ShiftRequest:} Solicitud de cambio de turno. Campos: \texttt{id}, \texttt{employee\_id}, \texttt{requested\_shift\_id}, \texttt{status} (PENDING/APPROVED/REJECTED), \texttt{requested\_at}, \texttt{response\_date}.
    
    \item \textbf{Product:} Artículo del catálogo de la sede. Campos: \texttt{id}, \texttt{sede\_id}, \texttt{nombre}, \texttt{precio}, \texttt{stock\_actual}, \texttt{stock\_minimo}.
    
    \item \textbf{Payroll:} Documento de nómina. Campos: \texttt{id}, \texttt{employee\_id}, \texttt{periodo\_mes}, \texttt{file\_url}, \texttt{uploaded\_at}.
    
    \item \textbf{Feedback:} Comentarios de empleados. Campos: \texttt{id}, \texttt{employee\_id}, \texttt{content}, \texttt{category} (BUG/SUGGESTION/COMPLAINT), \texttt{status} (OPEN/IN\_REVIEW/RESOLVED), \texttt{created\_at}.
\end{itemize}

% TODO: DIAGRAMA ENTIDAD-RELACIÓN
% \begin{figure}[h]
%     \centering
%     \includegraphics[width=0.9\textwidth]{imagenes/erd_diagram.png}
%     \caption{Diagrama Entidad-Relación: relaciones principales entre User, Sede, Shift, Product, Payroll, Feedback.}
%     \label{fig:erd}
% \end{figure}

\section{Casos de Uso Principales}

Esta sección enumera los casos de uso más críticos del sistema, estructurados por rol:

\subsection*{CEO/Administrador}
\begin{enumerate}
    \item \textbf{UC1: Consultar Dashboard Ejecutivo:} CEO accede a resumen de KPIs (ventas global, ocupación plantilla por sede). Sistema filtra datos por período (hoy, semana, mes). Datos se actualizan en tiempo real vía WebSockets.
    
    \item \textbf{UC2: Dar de Alta Nueva Sede:} CEO rellena formulario (nombre, ubicación, manager asignado). Sistema valida que el manager no esté asignado a otra sede. Crea registro en DB y envía email de bienvenida al manager.
    
    \item \textbf{UC3: Revisar Feedback de Empleados:} CEO accede a tablero de feedback filtrable por sede, categoría, estado. Puede pasar de OPEN a IN\_REVIEW y luego RESOLVED.
\end{enumerate}

\subsection*{Gerente de Sede}
\begin{enumerate}
    \item \textbf{UC4: Asignar Turno a Empleado:} Gerente accede a calendario semanal. Selecciona empleado y fecha. Sistema valida que el empleado no tenga turno solapado. Asigna turno con estado PENDING (requiere confirmación de empleado vía app).
    
    \item \textbf{UC5: Gestionar Solicitud de Permuta:} Dos empleados intercambian turnos. Uno solicita (genera ShiftRequest). Gerente aprueba o rechaza. Si aprueba, los turnos se actualizan. Si rechaza, sigue en estado PENDING.
    
    \item \textbf{UC6: Crear Producto en Catálogo:} Gerente rellena formulario (nombre, precio, stock inicial). Sistema valida que nombre no sea duplicado en la sede. Genera product\_id autoincremental.
    
    \item \textbf{UC7: Ajustar Stock:} Gerente selecciona producto, incremente/decrementa cantidad. Sistema genera log de movimiento (auditoría). Si stock $<$ stock\_minimo, genera alerta visual.
    
    \item \textbf{UC8: Subir Nóminas:} Gerente selecciona período de mes y archivo PDF. Sistema valida que sea PDF. Vincula a employees por nombre/email. Notifica a empleado que su nómina está disponible.
\end{enumerate}

\subsection*{Empleado}
\begin{enumerate}
    \item \textbf{UC9: Consultar Próximo Turno:} Empleado abre dashboard. Sistema muestra próximo turno (fecha, hora, sede). Si turno en <24h, destaca en rojo.
    
    \item \textbf{UC10: Solicitar Cambio de Turno:} Empleado selecciona turno actual y turno deseado. Sistema valida disponibilidad del turno deseado. Crea ShiftRequest con status=PENDING. Notifica al gerente.
    
    \item \textbf{UC11: Ver Historial de Nóminas:} Empleado accede a sección ``Mis Nóminas''. Sistema lista todos los períodos con PDF descargable. Filtrable por año/mes.
    
    \item \textbf{UC12: Enviar Feedback:} Empleado rellena formulario con comentario, categoría (BUG/SUGGESTION/COMPLAINT). Sistema asigna status=OPEN y timestamp. Feedback viaja hacia CEO (visible en UC3).
\end{enumerate}

% TODO: DIAGRAMA DE CASOS DE USO UML
% \begin{figure}[h]
%     \centering
%     \includegraphics[width=0.85\textwidth]{imagenes/use_case_diagram.png}
%     \caption{Diagrama de casos de uso UML: actores (CEO, Manager, Employee) y relaciones con casos de uso principal del sistema.}
%     \label{fig:use_cases}
% \end{figure}

\section{Definición de Perfiles y Privilegios}

El sistema implementa un modelo de seguridad basado en roles (RBAC) jerárquico que garantiza el principio de mínimo privilegio: cada usuario accede únicamente a la información y funcionalidades estrictamente necesarias para el desempeño de su labor.

A continuación, se detalla el alcance funcional y los subsistemas accesibles para cada uno de los tres roles definidos en la organización. Cabe destacar que \textbf{toda la interfaz ha sido diseñada para ser ``Responsive''}, permitiendo su uso fluido tanto en ordenadores de escritorio como en tablets y dispositivos móviles, adaptando la disposición de los elementos al tamaño de la pantalla disponible.

\subsection{Subsistema de Dirección (Rol: CEO/Superadministrador)}

Este perfil dispone de una visión panorámica y estratégica de la organización. Su funcionalidad está orientada a la supervisión y al análisis comparativo, sin necesidad de intervenir en la operativa diaria de cada local.

\subsubsection{Dashboard Ejecutivo}

Panel de control principal con métricas en tiempo real y accesos rápidos. Actualización automática cada 30 segundos via WebSockets (sin necesidad de refrescar).

\begin{itemize}
    \item \textbf{Sección de Rendimiento (Arriba izquierda):}
    \begin{itemize}
        \item Gráfica de líneas: Ingresos acumulados (global) vs. período anterior. Descargable como CSV/PNG.
        \item Tabla comparativa de sedes: Ingresos, ocupación plantilla (\%), stock total. Filtrable y ordenable.
        \item Top 5 productos más vendidos (gráfica de barras) y productos con bajo stock (alerta visual).
    \end{itemize}
    
    \item \textbf{Sección de RRHH (Arriba derecha):}
    \begin{itemize}
        \item ``Headcount'': Total empleados activos vs. bajas (sick leave, vacation). Indicador de % de ocupación.
        \item Tablero de ausencias: Qué empleados están de baja hoy (click para detalles).
        \item Turnover rate: \% de rotación en últimos 12 meses.
    \end{itemize}
    
    \item \textbf{Atajos de Gestión (Abajo):}
    \begin{itemize}
        \item Botón para crear nueva sede (modal form).
        \item Listado de gerentes con contacto rápido (teléfono, email).
    \end{itemize}
\end{itemize}

\textbf{Validaciones y Reglas:} Datos mostrados siempre corresponden a 24 horas atrás (lag de 1 día) para mantener consistencia. CEO solo ve datos de sedes donde tiene autorización.

% TODO: MOCKUP DASHBOARD CEO
% \begin{figure}[h]
%     \centering
%     \includegraphics[width=0.95\textwidth]{imagenes/mockup_ceo_dashboard.png}
%     \caption{Mockup del Dashboard Ejecutivo: KPIs de rendimiento, RRHH y atajos de gestión. Diseño responsivo.}
%     \label{fig:mockup_ceo}
% \end{figure}

\subsubsection{Gestión de Infraestructura (Sedes)}

\textbf{Crear Nueva Sede:} Formulario con campos:
\begin{itemize}
    \item Nombre (requerido, máx 100 caracteres)
    \item Ubicación/dirección (requerida, máx 200 caracteres)
    \item Manager (dropdown de usuarios con rol MANAGER no asignados)
    \item Teléfono de contacto (opcional, formato internacional)
\end{itemize}

\textbf{Validaciones:}
\begin{itemize}
    \item El manager seleccionado no debe estar ya asignado a otra sede.
    \item Nombre de sede debe ser único dentro de la organización.
    \item Teléfono se valida con librería \texttt{libphonenumber-js}.
\end{itemize}

Al crear, sistema genera email de notificación al manager con credenciales de acceso y link a primer login.

\textbf{Editar/Ver Sedes:} Tabla con todas las sedes (2026 del año actual). Filtrable por nombre, manager. Expandible para ver detalles: empleados activos, productos, ventas del mes.

\subsubsection{Auditoría de Feedback}

Tablero consolidado de comentarios constructivos de empleados hacia CEO.

\begin{itemize}
    \item \textbf{Listado de Feedback:} Tabla con columnas: Empleado, Categoría (BUG/SUGGESTION/COMPLAINT), Estado (OPEN/IN\_REVIEW/RESOLVED), Fecha creación.
    \item \textbf{Filtros:} Por sede, categoría, estado, rango de fechas.
    \item \textbf{Detalles:} Click en fila abre modal con texto completo, posibilidad de cambiar estado.
\end{itemize}

\textbf{Estados y Transiciones:} OPEN → IN\_REVIEW (CEO marca como leído) → RESOLVED (CEO cierra después de accionar). Empleado notificado cuando su feedback pasa a RESOLVED.

\subsection{Subsistema de Gestión Local (Rol: Gerente de Sede)}

Es el perfil con mayor carga de interacción en el sistema. Gestiona los recursos humanos y materiales de \textbf{su ubicación asignada exclusivamente}. Todas las operaciones están filtradas por \texttt{sede\_id = gerente.sede\_id}.

\subsubsection{Dashboard del Gerente}

Panel operativo optimizado para decisiones rápidas.

\begin{itemize}
    \item \textbf{Resumen del Día Actual:}
    \begin{itemize}
        \item Empleados presentes hoy (foto, nombre, hora entrada esperada).
        \item Alertas urgentes: bajas imprevistas, bajo stock crítico, solicitudes pendientes de aprobación.
        \item Botones de acceso rápido: ``Asignar Turno'', ``Aprobar Permuta'', ``Crear Producto''.
    \end{itemize}
    
    \item \textbf{Mini-gráficas:}
    \begin{itemize}
        \item Ocupación plantilla esta semana (\% promedio).
        \item Stock total de productos (alerta si <20\% del mínimo).
        \item Ingresos del mes (comparativa con mes anterior).
    \end{itemize}
\end{itemize}

% TODO: MOCKUP DASHBOARD GERENTE
% \begin{figure}[h]
%     \centering
%     \includegraphics[width=0.95\textwidth]{imagenes/mockup_manager_dashboard.png}
%     \caption{Mockup Dashboard Gerente: resumen operativo del día, alertas y accesos rápidos a funciones más usadas.}
%     \label{fig:mockup_manager}
% \end{figure}

\subsubsection{Gestión de Plantilla (CRUD Empleados)}

\textbf{Alta de Empleado:} Formulario con campos:
\begin{itemize}
    \item Nombre completo, email, teléfono (requeridos)
    \item Puesto (dropdown: Vendedor, Supervisor, Operario)
    \item Horas contratadas/semana (requerido, rango 10-40)
    \item Fecha de inicio (requerida, mín. hoy)
\end{itemize}

\textbf{Validaciones:}
\begin{itemize}
    \item Email debe ser único en la organización.
    \item Horas contratadas entre 10 y 40 horas/semana.
    \item Sistema genera \texttt{user\_id} autoincremental y contraseña temporal (enviada por email).
\end{itemize}

\textbf{Edición:} Permite cambiar puesto, horas contratadas. Cambios auditorados en log (quién, cuándo, qué cambió).

\textbf{Baja/Inactivación:} Marca empleado como \texttt{status=INACTIVE}. Su sueldo deja de acumularse desde esa fecha. Turnos futuros se liberan.

\subsubsection{Planificador y Gestión de Turnos}

\textbf{Flujo de Asignación de Turno (UC4):}

\begin{enumerate}
    \item Gerente accede a vista de calendario semanal/mensual (drag-and-drop).
    \item Selecciona un empleado y un día.
    \item Especifica hora entrada y hora salida (ambas requeridas, no pueden ser iguales).
    \item Sistema valida:
    \begin{itemize}
        \item Empleado está activo (status=ACTIVE).
        \item No hay turno solapado (overlap detection): \texttt{(new.start < existing.end) AND (new.end > existing.start)}.
        \item Turno no va más allá de horas contratadas/semana.
    \end{itemize}
    \item Si validaciones pasan: turno se crea con estado \texttt{status=PENDING}.
    \item Empleado recibe notificación push/email: ``Nuevo turno asignado [fecha, horas]. Confírma tu asistencia''.
\end{enumerate}

\textbf{Estados de Turno:}
\begin{itemize}
    \item \texttt{PENDING}: Recién asignado, espera confirmación de empleado.
    \item \texttt{CONFIRMED}: Empleado confirmó asistencia.
    \item \texttt{WORKED}: Turno ya pasó (fechado en el pasado).
    \item \texttt{CANCELLED}: Anulado (por gerente o empleado con causa).
    \item \texttt{NO\_SHOW}: Empleado no se presentó sin justificación.
\end{itemize}

\textbf{Vista Diaria:} Tabla con turnos de hoy, mostrando: Empleado, Hora entrada/salida, Estado, Acciones (editar/cancelar si aún no confirmado).

% TODO: FLOWCHART DE ASIGNACIÓN DE TURNO
% \begin{figure}[h]
%     \centering
%     \includegraphics[width=0.7\textwidth]{imagenes/flowchart_shift_assignment.png}
%     \caption{Flujo de validación y asignación de turno: detección de solapamientos, validación de horas, cambio de estado.}
%     \label{fig:flowchart_shift}
% \end{figure}

\textbf{Solicitudes de Permuta (UC5):}

Dos empleados quieren intercambiar turnos.

\begin{enumerate}
    \item Empleado A solicita permuta desde su app: selecciona su turno (T1) y propone intercambiar con turno de B (T2).
    \item Sistema valida: B no tiene conflicto si toma T1. A no tiene conflicto si toma T2.
    \item Crea \texttt{ShiftRequest} con \texttt{status=PENDING}.
    \item Gerente ve en ``Centro de Notificaciones'' solicitud de permuta. Puede:
    \begin{itemize}
        \item \textbf{Aprobar:} Turnos se intercambian. Ambos empleados notificados.
        \item \textbf{Rechazar:} ShiftRequest.status = REJECTED. Empleados notificados.
    \end{itemize}
\end{enumerate}

\textbf{Centro de Notificaciones:} Panel desplegable con:
\begin{itemize}
    \item Solicitudes de permutas pendientes (con avatares de empleados).
    \item Solicitudes de baja/permiso (si implementado).
    \item Alertas de operación (ej: empleado confirmó ausencia, nuevo empleado dado de alta).
\end{itemize}

\subsubsection{Gestión de Catálogo e Inventario}

\textbf{CRUD de Productos (UC6):}

Crear: Formulario con Nombre, Descripción (opcional), Precio venta, Cantidad inicial. Sistema valida nombre único por sede. Genera \texttt{product\_id}.

Editar: Permite cambiar precio y descripción. Cambios se auditorean.

Eliminar: Marca \texttt{status=DELETED} (soft delete). No aparece en catálogo pero se preserva en histórico de ventas.

% TODO: MOCKUP CATÁLOGO PRODUCTOS
% \begin{figure}[h]
%     \centering
%     \includegraphics[width=0.85\textwidth]{imagenes/mockup_catalog.png}
%     \caption{Mockup sección de Catálogo: tabla de productos con precios, stock, acciones (edit/delete).}
%     \label{fig:mockup_catalog}
% \end{figure}

\textbf{Control de Stock (UC7):}

\begin{itemize}
    \item \textbf{Ajuste Manual:} Gerente selecciona producto, ingresa cantidad a ajustar (+ o -). Sistema genera log: \texttt{Ajuste: [producto], [cantidad], [motivo], [usuario], [timestamp]}.
    
    \item \textbf{Alertas de Stock:} Si \texttt{stock\_actual < stock\_minimo}, icono de alerta en rojo en dashboard. Gerente puede reordenar (funcionalidad futura de integración POS).
    
    \item \textbf{Inventario:} Opción ``Realizar Inventario'': Gerente cuenta productos físicos y ajusta cantidades en sistema (puede tomar foto como comprobante).
\end{itemize}

\textbf{Validaciones:}
\begin{itemize}
    \item Stock nunca puede ser negativo.
    \item Ajustes requieren motivo (selección de lista dropdown: Venta, Rotura, Aportación, Corrección, etc.).
\end{itemize}

\subsubsection{Gestión Documental (Nóminas)}

\textbf{Subida de Nóminas (UC8):}

\begin{enumerate}
    \item Gerente accede a sección ``Nóminas''.
    \item Selecciona período (mes/año de dropdown).
    \item Sube archivo PDF. Sistema valida:
    \begin{itemize}
        \item Es PDF (no otra extensión).
        \item Tamaño <10MB.
    \end{itemize}
    \item Gerente selecciona de dropdown qué empleado es destinatario (o sube múltiples nóminas a la vez con asignación automática por nombre).
    \item Sistema crea registro \texttt{Payroll} e inmediatamente notifica al empleado: ``Tu nómina de [mes] está disponible''.
\end{enumerate}

\textbf{Auditoría:} Cada PDF se almacena encriptado (AES-256) en almacenamiento seguro. Acceso se loguea.

\subsection{Subsistema Operativo (Rol: Empleado)}

Diseñado con una interfaz simplificada y \textit{Mobile-First}, este perfil actúa como un consumidor de información y emisor de feedback. Todas las pantallas se cargan en <2 segundos en conexión 3G.

\subsubsection{Dashboard del Empleado}

Pantalla inicial (home screen) personalizada para cada empleado.

\begin{itemize}
    \item \textbf{Tarjeta de Próximo Turno (Destacada):}
    \begin{itemize}
        \item Fecha, hora entrada/salida de turno más próximo.
        \item Si turno es hoy: botón ``Confirmar Asistencia'' (debe presionarlo antes del turno).
        \item Si turno es en <24h: se destaca en amarillo. Si es en <1h: en rojo.
        \item Botón ``+ Info'': detalles de ubicación sede, contacto gerente.
    \end{itemize}
    
    \item \textbf{Notificaciones Relevantes:}
    \begin{itemize}
        \item Nuevo turno asignado (gerente lo acaba de crear).
        \item Permuta aprobada/rechazada.
        \item Nueva nómina disponible.
    \end{itemize}
    
    \item \textbf{Acceso Rápido:}
    \begin{itemize}
        \item Botón a ``Mis Turnos'' (calendario).
        \item Botón a ``Mis Nóminas''.
        \item Botón a ``Enviar Feedback''.
    \end{itemize}
\end{itemize}

% TODO: MOCKUP DASHBOARD EMPLEADO MÓVIL
% \begin{figure}[h]
%     \centering
%     \includegraphics[width=0.5\textwidth]{imagenes/mockup_employee_dashboard_mobile.png}
%     \caption{Mockup Dashboard Empleado (móvil): próximo turno, notificaciones, accesos rápidos. Optimizado para pantalla pequeña.}
%     \label{fig:mockup_employee}
% \end{figure}

\subsubsection{Calendario y Turnos}

\textbf{Vistas de Calendario (UC9):}

\begin{itemize}
    \item \textbf{Vista Semanal (default):} Grilla con 7 días. Cada celda muestra turno (si existe) con hora. Color verde = CONFIRMED, amarillo = PENDING, gris = WORKED (pasado).
    
    \item \textbf{Vista Mensual:} Calendario clásico. Cada día con pequeño indicador (punto) si tiene turno. Click en día expande para ver detalles.
    
    \item \textbf{Filtros:} Sede (si aplica), estado de turno.
\end{itemize}

\textbf{Solicitud de Cambio de Turno (UC10):}

Flujo de permuta desde perspectiva de empleado:

\begin{enumerate}
    \item Empleado selecciona su turno (T1) desde calendario. Aparece botón ``Solicitar Permuta''.
    \item Ventana emerge permitiendo seleccionar turno de otro compañero (T2). Sistema muestra solo turnos de compañeros disponibles ese día (búsqueda filtrada).
    \item Empleado confirma. Sistema valida que ambos turnos sean compatibles.
    \item \texttt{ShiftRequest} se crea con \texttt{status=PENDING}. Aparece en notificaciones de empleado como ``En Espera de Aprobación''.
    \item Mientras solicitud esté pending, no puede refundir new requests (evita spam).
    \item Cuando gerente aprueba/rechaza, empleado recibe notificación push inmediata.
\end{enumerate}

\subsubsection{Mis Nóminas}

\textbf{Visualización y Descarga (UC11):}

\begin{itemize}
    \item \textbf{Listado de Nóminas:} Tabla con columnas: Período (Ej: ``Enero 2026''), Fecha disponible, Acciones (Descargar).
    \item \textbf{Filtro por Año:} Dropdown para cambiar año y ver histórico completo.
    \item \textbf{Descargar:} Click en botón descarga PDF (encriptado en almacenamiento, se descifra al descargar).
    \item \textbf{Seguridad:} Empleado SOLO ve sus propias nóminas. Queryse filtra por \texttt{user\_id = logged\_user.id}.
\end{itemize}

\subsubsection{Sistema de Feedback}

\textbf{Envío de Feedback (UC12):}

\begin{itemize}
    \item \textbf{Formulario:} Campos:
    \begin{itemize}
        \item ``¿De qué es tu comentario?'' (radio button): BUG / SUGGESTION / COMPLAINT.
        \item ``Tu mensaje'' (textarea, máx 500 caracteres).
    \end{itemize}
    
    \item \textbf{Validaciones:}
    \begin{itemize}
        \item Mensaje no puede estar vacío.
        \item Máximo 3 feedbacks por día (throttling para evitar spam).
    \end{itemize}
    
    \item \textbf{Al Enviar:}
    \begin{itemize}
        \item Sistema crea registro \texttt{Feedback} con \texttt{status=OPEN}.
        \item Toast de confirmación: ``Gracias por tu feedback. Lo recibirá nuestro equipo''.
        \item CEO lo ve en su tablero de Auditoría (UC3).
    \end{itemize}
\end{itemize}

% TODO: MOCKUP FORMULARIO FEEDBACK
% \begin{figure}[h]
%     \centering
%     \includegraphics[width=0.55\textwidth]{imagenes/mockup_feedback_form.png}
%     \caption{Mockup Formulario de Feedback: simple, mobile-friendly, claro.}
%     \label{fig:mockup_feedback}
% \end{figure}

\section{Sistema de Notificaciones y Alertas}

Las notificaciones son críticas para mantener usuarios informados en tiempo real.

\subsection*{Canales de Notificación}

\begin{itemize}
    \item \textbf{Push Notifications (In-App):} Aparecen en esquina superior de pantalla. Auto-desaparecen en 5 segundos. Click lleva a pantalla relevante.
    
    \item \textbf{Email:} Eventos importantes (nuevo turno, permuta rechazada, nómina disponible). Enviados vía servicio SMTP (SendGrid o similar). \textbf{No spam}: máximo 1 email por día por tipo de evento.
    
    \item \textbf{En-App Badge:} Ícono de campana con contador (ej: ``3 notificaciones nuevas'').
\end{itemize}

\subsection*{Eventos que Generan Notificaciones}

\begin{table}[h]
    \centering
    \caption{Matriz de notificaciones: qué rol recibe qué evento por qué canal}
    \label{tab:notifications}
    \vspace{0.3cm}
    \begin{tabularx}{\textwidth}{l l X}
        \toprule
        \textbf{Evento} & \textbf{Rol} & \textbf{Canales} \\
        \midrule
        Nuevo turno asignado & Empleado & Push + Email \\
        Turno confirmado/rechazado & Gerente & Push \\
        Solicitud de permuta recibida & Gerente & Push \\
        Permuta aprobada/rechazada & Empleado & Push + Email \\
        Nómina disponible & Empleado & Push + Email \\
        Nuevo feedback recibido & CEO & Push \\
        Stock bajo crítico & Gerente & Push \\
        Nueva sede creada & Gerente (assigned) & Email \\
        \bottomrule
    \end{tabularx}
\end{table}

\section{Flujos de Navegación y Menú}

Cada rol dispone de un menú lateral (en desktop) o hamburguesa (en móvil) con secciones accesibles.

\subsection*{CEO}
\begin{itemize}
    \item Dashboard Ejecutivo
    \item Gestión de Sedes
    \item Auditoría de Feedback
    \item Configuración (cambiar contraseña, idioma, zona horaria)
    \item Logout
\end{itemize}

\subsection*{Gerente}
\begin{itemize}
    \item Dashboard Operativo
    \item Planificador de Turnos
    \item Gestión de Empleados
    \item Catálogo e Inventario
    \item Nóminas
    \item Centro de Notificaciones
    \item Mi Perfil
    \item Logout
\end{itemize}

\subsection*{Empleado}
\begin{itemize}
    \item Mis Turnos (Calendario)
    \item Mis Nóminas
    \item Enviar Feedback
    \item Mi Perfil (editar teléfono, email, contraseña)
    \item Logout
\end{itemize}

\section{Integraciones y Sincronización en Tiempo Real}

\subsection*{Sincronización de Datos Entre Capas}

Las operaciones críticas emplean \textbf{transacciones ACID} en PostgreSQL para garantizar consistencia:

\begin{itemize}
    \item \textbf{Al asignar turno:} En una única transacción se actualiza \texttt{Shift.status}, se decrementa horas disponibles del empleado, se incrementa una columna de auditoría. Si falla cualquier paso, ROLLBACK.
    
    \item \textbf{Al aprobar permuta:} Transacción que intercambia IDs de \texttt{shift.employee\_id} de ambos turnos de forma atómica. Imposible que quede estado inconsistente.
\end{itemize}

\subsection*{Actualizaciones en Tiempo Real para CEO}

El Dashboard Ejecutivo se actualiza automáticamente sin que CEO necesite refrescar:

\begin{itemize}
    \item Backend emite eventos vía WebSocket cuando cambian datos relevantes: nuevo turno creado, nómina subida, feedback enviado.
    \item Frontend mantiene conexión WebSocket abierta. Al recibir evento, actualiza gráficas y tablas (sin molestar si CEO está interactuando).
    \item Fallback: Si WebSocket se cae, frontend hace polling cada 30 segundos.
\end{itemize}

\section{Privacidad y Consideraciones de Seguridad}

\subsection*{Filtrado de Datos}

\textbf{Principio:} Cada usuario SOLO ve datos según su rol:

\begin{itemize}
    \item \textbf{Empleado:} Ve solo sus propios turnos, su nómina, su feedback.
    \item \textbf{Gerente:} Ve empleados, turnos, inventario de su sede solamente. No ve datos de otras sedes.
    \item \textbf{CEO:} Ve agregados (suma, promedio) pero no datos individuales (ej: no ve nombre de empleado en turno, solo código employeeID si necesario).
\end{itemize}

\textbf{Implementación:} Middleware Express valida cada query: \texttt{WHERE sede\_id = user.sede\_id OR user.role = ADMIN}. Queries SQL parameterizadas previenen inyecciones.

\subsection*{Protección de Datos Sensibles}

\begin{itemize}
    \item \textbf{Nóminas:} PDFs encriptados (AES-256) en almacenamiento. Descifrado solo en lado cliente, nunca transmitido en claro.
    \item \textbf{Contraseñas:} Hasheadas con bcrypt (10 rounds). No se almacena ni se recupera texto plano.
    \item \textbf{Datos de contacto (email/teléfono):} Accesibles solo a roles autorizados (gerente ve teléfono de sus empleados, CEO no).
    \item \textbf{HTTPS obligatorio:} Todas las comunicaciones con TLS 1.2+. HSTS header configurado.
\end{itemize}

\subsection*{Auditoría}

Operaciones críticas se registran en tabla \texttt{AuditLog}:

\begin{itemize}
    \item Quién (user\_id), Qué (operación), Cuándo (timestamp), Dónde (tabla afectada).
    \item Ejemplos: ``Manager user\_42 creó turno para employee\_15, 2026-02-09 10:30'', ``emp\_18 descaró nómina enero, 2026-02-09 14:45''.
    \item Acceso a auditlog restringido a CEO (solo para su sede).
\end{itemize}

\section{Testing de Funcionalidades Críticas}

La especificación funcional anterior debe validarse mediante tests:

\subsection*{Unit Tests Recomendados}
\begin{itemize}
    \item Validación de overlap de turnos (función pura).
    \item Cálculo de horas totales/semana.
    \item Reglas de permisos RBAC.
\end{itemize}

\subsection*{Integration Tests Recomendados}
\begin{itemize}
    \item Flujo completo asignación turno: crear → validar → guardar → notificar.
    \item Flujo aprobación permuta: ambos turnos intercambian correctamente.
    \item Subida nómina: archivo guardado, empleado notificado, accesible en app.
\end{itemize}

\subsection*{E2E Tests Recomendados}
\begin{itemize}
    \item CEO login → ver dashboard → crear sede → verificar aparece.
    \item Gerente login → asignar turno → empleado ve notificación.
    \item Empleado solicita permuta → gerente aprueba → ambos ven cambios.
\end{itemize}

\section{Consideraciones para Futuras Extensiones}

El diseño modular permite agregar funcionalidades sin afectar lo existente:

\begin{itemize}
    \item \textbf{Sistema de Calificaciones:} CEO y Gerentes puntúan empleados (1-5 estrellas). Permite reestructuración de equipos.
    \item \textbf{Integración POS:} Conectar con caja registradora para registrar ventas automáticamente (hoy es manual).
    \item \textbf{Horas Extras y Suplementos:} Cálculos de compensación si turno excede horas contratadas.
    \item \textbf{Roles Adicionales:} Supervisor, Coordinador (entre Manager y CEO).
    \item \textbf{Aplicación Móvil Nativa:} Mismo backend API, cliente iOS/Android con React Native.
    \item \textbf{Integración HR Compliance:} Conectar con software de recursos humanos (Factorial, BizMérida) para sincronizar empleados.
\end{itemize}
