% ==========================================================
% CAPÍTULO 5: PLAN DE DESARROLLO ÁGIL Y DIVISIÓN DE TAREAS
% ==========================================================
\chapter{Plan de Desarrollo Ágil}
\label{ch:desarrollo}

\section{Metodología de Desarrollo}

El proyecto se desarrolla bajo \textbf{metodología Scrum adaptada} con ciclos iterativos de 2 semanas.

\subsection*{Parámetros del Proyecto}

\begin{itemize}
    \item \textbf{Duración Total:} 12 semanas (febrero 9 - mayo 4, 2026)
    \item \textbf{Dedicación:} 15 horas/semana (promedio)
    \item \textbf{Total de Horas Disponibles:} 180 horas
    \item \textbf{Número de Sprints:} 6 sprints de 2 semanas
    \item \textbf{Horas/Sprint:} ~30 horas (2 sprints paralelos = 60h/mes)
\end{itemize}

\section{Product Backlog Priorizado}

El Product Backlog contiene todas las funcionalidades organizadas por prioridad, estimadas en \textbf{Story Points}.


\subsection*{Grupo 1: Infraestructura y Seguridad (CRÍTICO - Sprint 1-2)}

\begin{table}[h]
    \centering
    \caption{Backlog: Infraestructura}
    \vspace{0.3cm}
    \begin{tabularx}{\textwidth}{l l c l}
        \toprule
        \textbf{ID} & \textbf{Descripción} & \textbf{SP} & \textbf{Prioridad} \\
        \midrule
        PBI-1 & Setup: git, ESLint, Prettier, carpetas & 3 & ALTA \\
        PBI-2 & PostgreSQL + schema inicial (User, Sede, Shift) & 3 & ALTA \\
        PBI-3 & Express: servidor, middleware, CORS & 2 & ALTA \\
        PBI-4 & React + Vite + Tailwind + componentes base & 3 & ALTA \\
        PBI-5 & JWT: access/refresh tokens + logout & 5 & ALTA \\
        PBI-6 & Endpoint /auth/login + validaciones & 3 & ALTA \\
        PBI-7 & Pantalla Login (frontend) & 3 & ALTA \\
        PBI-8 & RBAC middleware (CEO/Manager/Employee) & 5 & ALTA \\
        PBI-9 & Endpoint /auth/register (solo admin) & 2 & ALTA \\
        \bottomrule
    \end{tabularx}
\end{table}

\subsection*{Grupo 2: Gestión de Usuarios (Sprint 2-3)}

\begin{table}[h]
    \centering
    \caption{Backlog: Usuarios y Sedes}
    \vspace{0.3cm}
    \begin{tabularx}{\textwidth}{l l c l}
        \toprule
        \textbf{ID} & \textbf{Descripción} & \textbf{SP} & \textbf{Prioridad} \\
        \midrule
        PBI-10 & Endpoints CRUD usuarios (GET/POST/PUT/DELETE) & 5 & ALTA \\
        PBI-11 & Endpoints CRUD sedes & 3 & ALTA \\
        PBI-12 & Validaciones Zod (usuario, sede) & 2 & ALTA \\
        PBI-13 & Pantalla Admin: Crear usuario + asignar rol/sede & 5 & ALTA \\
        PBI-14 & Pantalla Admin: Listar usuarios, filtrar, editar & 5 & ALTA \\
        \bottomrule
    \end{tabularx}
\end{table}

\subsection*{Grupo 3: Dashboard CEO (Sprint 3-4)}

\begin{table}[h]
    \centering
    \caption{Backlog: Dashboard Ejecutivo}
    \vspace{0.3cm}
    \begin{tabularx}{\textwidth}{l l c l}
        \toprule
        \textbf{ID} & \textbf{Descripción} & \textbf{SP} & \textbf{Prioridad} \\
        \midrule
        PBI-15 & Endpoint GET /dashboard/metrics (ventas, ocupación, stock) & 8 & ALTA \\
        PBI-16 & Pantalla Dashboard: layout KPIs básico & 5 & ALTA \\
        PBI-17 & Gráficas Recharts (líneas, barras, tablas) & 5 & MEDIA \\
        PBI-18 & TanStack Query: caché de datos en frontend & 3 & MEDIA \\
        PBI-19 & Filtros por fecha y sede & 3 & MEDIA \\
        \bottomrule
    \end{tabularx}
\end{table}

\subsection*{Grupo 4: Gestión de Turnos (Sprint 3-5) - CORE}

\begin{table}[h]
    \centering
    \caption{Backlog: Turnos}
    \vspace{0.3cm}
    \begin{tabularx}{\textwidth}{l l c l}
        \toprule
        \textbf{ID} & \textbf{Descripción} & \textbf{SP} & \textbf{Prioridad} \\
        \midrule
        PBI-20 & Endpoints CRUD turnos (crear, listar, cancelar) & 8 & ALTA \\
        PBI-21 & Validación overlaps + unit tests & 5 & ALTA \\
        PBI-22 & Endpoint confirmar turno (empleado) & 3 & ALTA \\
        PBI-23 & Calendario Gerente: semanal + crear turno & 8 & ALTA \\
        PBI-24 & Calendario Empleado: semanal/mensual & 5 & ALTA \\
        \bottomrule
    \end{tabularx}
\end{table}

\subsection*{Grupo 5: Permutas de Turnos (Sprint 5)}

\begin{table}[h]
    \centering
    \caption{Backlog: Solicitudes de Permuta}
    \vspace{0.3cm}
    \begin{tabularx}{\textwidth}{l l c l}
        \toprule
        \textbf{ID} & \textbf{Descripción} & \textbf{SP} & \textbf{Prioridad} \\
        \midrule
        PBI-25 & Endpoint crear/listar ShiftRequest & 5 & MEDIA \\
        PBI-26 & Endpoint aprobar/rechazar permuta & 5 & MEDIA \\
        PBI-27 & Pantalla Gerente: Centro de Notificaciones & 5 & MEDIA \\
        PBI-28 & Pantalla Empleado: Solicitar permuta & 5 & MEDIA \\
        \bottomrule
    \end{tabularx}
\end{table}

\subsection*{Grupo 6: Inventario (Sprint 5-6)}

\begin{table}[h]
    \centering
    \caption{Backlog: Gestión de Productos}
    \vspace{0.3cm}
    \begin{tabularx}{\textwidth}{l l c l}
        \toprule
        \textbf{ID} & \textbf{Descripción} & \textbf{SP} & \textbf{Prioridad} \\
        \midrule
        PBI-29 & Endpoints CRUD productos (crear, editar, listar) & 5 & MEDIA \\
        PBI-30 & Endpoint ajustar stock + log cambios & 5 & MEDIA \\
        PBI-31 & Alertas bajo stock (lógica backend) & 2 & MEDIA \\
        PBI-32 & Pantalla Gerente: Catálogo productos & 5 & MEDIA \\
        PBI-33 & Pantalla Gerente: Ajustar stock & 3 & MEDIA \\
        \bottomrule
    \end{tabularx}
\end{table}

\subsection*{Grupo 7: Nóminas (Sprint 6 - OPCIONAL)}

\begin{table}[h]
    \centering
    \caption{Backlog: Gestión de Nóminas}
    \vspace{0.3cm}
    \begin{tabularx}{\textwidth}{l l c l}
        \toprule
        \textbf{ID} & \textbf{Descripción} & \textbf{SP} & \textbf{Prioridad} \\
        \midrule
        PBI-34 & Endpoint subir PDF nómina (validación básica) & 5 & BAJA \\
        PBI-35 & Endpoint listar nóminas empleado & 2 & BAJA \\
        PBI-36 & Pantalla Gerente: Subir nóminas & 3 & BAJA \\
        PBI-37 & Pantalla Empleado: Mis nóminas + descarga & 3 & BAJA \\
        \bottomrule
    \end{tabularx}
\end{table}


\subsection*{Grupo 8: Feedback y Testing (Sprint 6 - OPCIONAL)}

\begin{table}[h]
    \centering
    \caption{Backlog: Feedback y Testing}
    \vspace{0.3cm}
    \begin{tabularx}{\textwidth}{l l c l}
        \toprule
        \textbf{ID} & \textbf{Descripción} & \textbf{SP} & \textbf{Prioridad} \\
        \midrule
        PBI-38 & Endpoint feedback (crear, listar, cambiar estado) & 5 & BAJA \\
        PBI-39 & Pantalla Empleado: Formulario feedback & 2 & BAJA \\
        PBI-40 & Pantalla CEO: Tablero feedback & 3 & BAJA \\
        PBI-41 & Unit tests: 50+ tests, 80\% cobertura & 13 & BAJA \\
        PBI-42 & Integration tests: 5+ flujos críticos & 8 & BAJA \\
        \bottomrule
    \end{tabularx}
\end{table}

\section{Planificación de Sprints (6 x 2 semanas)}

\subsection*{Sprint 1 (Feb 9-22): Infraestructura y Autenticación}
\textbf{PBIs:} 1-9 (28 SP) \textbf{COMPLETADO}

\subsubsection*{PBI-1: Setup (git, ESLint, Prettier)}
Inicialización del repositorio con estructura de carpetas para frontend y backend. Se configuraron:
\begin{itemize}
    \item Repositorio Git en GitHub (TFG-Multihub)
    \item ESLint con configuración airbnb-typescript
    \item Prettier para formato automático
    \item Estructura monorepo: \texttt{/backend} y \texttt{/frontend}
\end{itemize}

\subsubsection*{PBI-2: PostgreSQL + Schema}
Base de datos con Prisma ORM. Schema incluye:
\begin{itemize}
    \item \textbf{User}: Modelo de empleados con roles (ADMIN, MANAGER, EMPLOYEE)
    \item \textbf{Sede}: Oficinas con manager asignado
    \item \textbf{Shift}: Turnos laborales con validación de no solapamiento
    \item \textbf{ShiftRequest}: Solicitudes de permuta entre empleados
\end{itemize}

Ejemplo del schema en Prisma:
\begin{lstlisting}[language=typescript]
model User {
  id        String   @id @default(cuid())
  email     String   @unique
  password  String
  name      String
  role      UserRole
  sedeId    String?
  // ... relaciones ...
}

enum UserRole {
  ADMIN
  MANAGER
  EMPLOYEE
}
\end{lstlisting}

\subsubsection*{PBI-3: Express + Middleware}
Backend Node.js con Express configurado con:
\begin{itemize}
    \item Middleware CORS con credenciales habilitadas
    \item Body-parser para JSON
    \item Gestión de errores global
    \item Health check endpoint en \texttt{GET /health}
\end{itemize}

\subsubsection*{PBI-4: React + Vite + Tailwind}
Frontend con stack moderno:
\begin{itemize}
    \item React 18 con Vite como bundler
    \item Tailwind CSS para estilos
    \item React Router para navegación
    \item TanStack Query para caché de datos
    \item TypeScript strict mode
\end{itemize}

\subsubsection*{PBI-5: JWT Tokens}
Sistema de autenticación con JWT. Implementación de dos tokens:

\textbf{Access Token:} Duraión 15 minutos
\textbf{Refresh Token:} Duración 7 días (almacenado en httpOnly cookie)

Código de generación:
\begin{lstlisting}[language=typescript]
export function generateTokens(userId: string, email: string, role: string) {
  const accessToken = jwt.sign(
    { userId, email, role },
    process.env.JWT_SECRET!,
    { expiresIn: '15m' }
  );

  const refreshToken = jwt.sign(
    { userId, email, role },
    process.env.JWT_REFRESH_SECRET!,
    { expiresIn: '7d' }
  );

  return { accessToken, refreshToken };
}
\end{lstlisting}

Middleware de validación:
\begin{lstlisting}[language=typescript]
export const authMiddleware = (req: Request, res: Response, next: NextFunction) => {
  const token = req.headers.authorization?.split(' ')[1];

  if (!token) {
    return res.status(401).json({ error: 'No token provided' });
  }

  try {
    const payload = jwt.verify(token, process.env.JWT_SECRET!);
    req.user = payload;
    next();
  } catch (error) {
    res.status(401).json({ error: 'Invalid token' });
  }
};
\end{lstlisting}

\subsubsection*{PBI-6: Endpoint /auth/login}
Endpoint REST para autenticación:
\begin{itemize}
    \item Validación de credenciales contre base de datos
    \item Hash de contraseña con bcryptjs (10 rounds)
    \item Generación de JWT tokens
    \item Cookie httpOnly para refresh token
\end{itemize}

Código del servicio:
\begin{lstlisting}[language=typescript]
async login(credentials: LoginRequest): Promise<LoginResponse> {
  const validated = LoginSchema.parse(credentials);
  
  const user = await prisma.user.findUnique({
    where: { email: validated.email },
  });

  if (!user) {
    throw new Error('Usuario o contraseña incorrectos');
  }

  const passwordMatch = await bcryptjs.compare(
    validated.password, 
    user.password
  );

  if (!passwordMatch) {
    throw new Error('Usuario o contraseña incorrectos');
  }

  const tokens = generateTokens(user.id, user.email, user.role);
  return {
    user: this.excludePassword(user),
    tokens,
  };
}
\end{lstlisting}

\subsubsection*{PBI-7: Pantalla Login (Frontend)}
Componente React con:
\begin{itemize}
    \item Formulario con validaciones en tiempo real
    \item Mensajes de error específicos por campo
    \item Estado de carga con spinner animado
    \item Redirección automática a dashboard tras login exitoso
    \item Manejo de sesión con AuthContext
\end{itemize}

Almacenamiento de tokens:
\begin{lstlisting}[language=typescript]
export const getAccessToken = (): string | null => {
  return localStorage.getItem('accessToken');
};

export const setAccessToken = (token: string): void => {
  localStorage.setItem('accessToken', token);
};
\end{lstlisting}

AuthProvider para estado global:
\begin{lstlisting}[language=typescript]
export const AuthProvider: React.FC<{ children: React.ReactNode }> = ({ children }) => {
  const [user, setUser] = useState<User | null>(getStoredUser());

  const login = async (credentials: LoginRequest): Promise<void> => {
    const response = await authAPI.login(credentials);
    setUser(response.data.user);
    setAccessToken(response.data.accessToken);
  };

  const contextValue: AuthContextType = {
    user,
    isAuthenticated: !!user && !!getAccessToken(),
    login,
    logout,
    refreshTokens,
  };

  return React.createElement(AuthContext.Provider, { value: contextValue }, children);
};
\end{lstlisting}

\subsubsection*{PBI-8: RBAC Middleware}
Control de acceso basado en roles. Middleware para proteger endpoints:
\begin{lstlisting}[language=typescript]
export const roleMiddleware = (allowedRoles: UserRole[]) => {
  return (req: Request, res: Response, next: NextFunction) => {
    if (!req.user || !allowedRoles.includes(req.user.role)) {
      return res.status(403).json({ error: 'Access denied' });
    }
    next();
  };
};
\end{lstlisting}

Ejemplo de uso en rutas protegidas:
\begin{lstlisting}[language=typescript]
router.post('/register', 
  authMiddleware, 
  roleMiddleware(['ADMIN']), 
  (req, res) => authController.register(req, res)
);
\end{lstlisting}

\subsubsection*{PBI-9: Endpoint /auth/register}
Creación de usuarios por administrador. Características:
\begin{itemize}
    \item Contraseña temporal generada automáticamente (10 caracteres)
    \item Solo ADMIN puede crear usuarios
    \item El empleado cambia la contraseña en primer acceso
\end{itemize}

\begin{lstlisting}[language=typescript]
async register(
  data: RegisterRequest,
  adminId: string
): Promise<{ user: User; tempPassword: string }> {
  // Generar contraseña temporal (10 caracteres aleatorios)
  const tempPassword = Math.random().toString(36).slice(2, 12);
  const hashedPassword = await bcryptjs.hash(tempPassword, 10);

  const newUser = await prisma.user.create({
    data: {
      email: data.email,
      name: data.name,
      password: hashedPassword,
      role: data.role as UserRole,
      sedeId: data.sedeId,
    },
  });

  return {
    user: this.excludePassword(newUser),
    tempPassword,
  };
}
\end{lstlisting}

\subsubsection*{Validaciones con Zod}
Schema de validación para login y registro:
\begin{lstlisting}[language=typescript]
export const LoginSchema = z.object({
  email: z.string().email('Email inválido'),
  password: z.string().min(6, 'Contraseña mínimo 6 caracteres'),
});

export const RegisterSchema = z.object({
  email: z.string().email('Email inválido'),
  name: z.string().min(2, 'Nombre requerido'),
  role: z.enum(['ADMIN', 'MANAGER', 'EMPLOYEE']).default('EMPLOYEE'),
  sedeId: z.string().optional(),
});
\end{lstlisting}

\subsubsection*{Rutas Protegidas - Frontend}
Componente ProtectedRoute para control de acceso:
\begin{lstlisting}[language=typescript]
const ProtectedRoute: React.FC<ProtectedRouteProps> = ({ 
  children, 
  requiredRoles 
}) => {
  const { user, isAuthenticated } = useAuth();

  if (!isAuthenticated) {
    return <Navigate to="/login" replace />;
  }

  if (requiredRoles && !requiredRoles.includes(user?.role!)) {
    return <div>Acceso denegado</div>;
  }

  return <>{children}</>;
};
\end{lstlisting}

\subsection*{Sprint 1 - Resumen Técnico}
\begin{itemize}
    \item \textbf{Stack:} PERN + TypeScript
    \item \textbf{Autenticación:} JWT con access/refresh tokens
    \item \textbf{Base de datos:} PostgreSQL 16 con Prisma ORM
    \item \textbf{Validaciones:} Zod schemas
    \item \textbf{API Routes:} 7 endpoints de autenticación
    \item \textbf{Frontend:} React Router con rutas protegidas
    \item \textbf{DevOps:} Docker Compose, GitHub Actions CI/CD
\end{itemize}

\subsection*{Sprint 2 (Feb 23-Mar 8): RBAC y Gestión de Usuarios}
\textbf{PBIs:} 8-14 (30 SP)

\subsection*{Sprint 3 (Mar 9-22): Gestión de Turnos}
\textbf{PBIs:} 20-24 (28 SP)

\subsection*{Sprint 4 (Mar 23-Apr 5): Dashboard CEO}
\textbf{PBIs:} 15-19 (26 SP)

\subsection*{Sprint 5 (Apr 6-19): Permutas + Inventario}
\textbf{PBIs:} 25-33 (30 SP)

\subsection*{Sprint 6 (Apr 20-May 4): Nóminas, Feedback y Testing}
\textbf{PBIs:} 34-42 (30 SP)

