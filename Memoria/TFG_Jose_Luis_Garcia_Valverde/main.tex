\documentclass[12pt, a4paper, oneside]{report}

% ==========================================================
% PAQUETES DE CONFIGURACIÓN Y ESTILO
% ==========================================================

% Codificación e idioma
\usepackage[utf8]{inputenc}
\usepackage[T1]{fontenc}
\usepackage[spanish, es-tabla]{babel} 
\usepackage{tabularx}
% Tipografía principal: Palatino (serif) para cuerpo, Helvetica (sans) para títulos
\usepackage{mathpazo} 
\usepackage[scaled=0.95]{helvet}

% Geometría y Márgenes
\usepackage[a4paper, top=3cm, bottom=3cm, left=3cm, right=3cm]{geometry}

% Espaciado e interlineado
\usepackage{setspace}
\onehalfspacing 
\usepackage{parskip} 

% Gráficos y Tablas
\usepackage{graphicx}
\usepackage{booktabs} 
\usepackage{array}
\usepackage{multirow}
\usepackage{caption}
\usepackage{subcaption}
\captionsetup{font=small, labelfont=bf, margin=1cm} 

% Colores
\usepackage{xcolor}
% Definición de colores corporativos (UMU y generales)
\definecolor{umuRed}{RGB}{204, 0, 0}      % Rojo Institucional
\definecolor{umuBlue}{RGB}{0, 51, 102}    % Azul Institucional
\definecolor{primaryColor}{RGB}{0, 51, 102} % Usamos el azul como primario para headers
\definecolor{grayText}{RGB}{80, 80, 80}

% Bibliografía
\usepackage[backend=biber, style=ieee]{biblatex}
\addbibresource{bibliografia.bib}

% Enlaces interactivos
\usepackage[
    colorlinks=true,
    linkcolor=umuBlue,
    citecolor=umuRed,
    urlcolor=umuBlue,
    pdfauthor={José Luis García Valverde},
    pdftitle={TFG Gestión Empresarial Multi-Sede},
    pdfsubject={Trabajo Fin de Grado},
    pdfkeywords={ERP, React, Node.js, Multi-sede, RRHH}
]{hyperref}

% ==========================================================
% CÓDIGO Y LISTADOS
% ==========================================================

\usepackage{listings}
\lstset{
    language=java,
    basicstyle=\ttfamily\small,
    breaklines=true,
    keywordstyle=\color{umuBlue}\bfseries,
    commentstyle=\color{gray}\itshape,
    stringstyle=\color{red},
    numberstyle=\tiny\color{gray},
    showstringspaces=false,
    tabsize=2,
    frame=tb,
    rulecolor=\color{lightgray},
    backgroundcolor=\color{white},
    captionpos=b,
    numbersep=5pt,
    breakatwhitespace=false,
    escapeinside={\%*}{*)},
}

% ==========================================================
% DISEÑO DE CABECERAS Y TÍTULOS
% ==========================================================

\usepackage{fancyhdr}
\pagestyle{fancy}
\fancyhf{}
\fancyhead[L]{\small \sffamily \nouppercase{\leftmark}} 
\fancyhead[R]{\small \sffamily \thepage} 
\renewcommand{\headrulewidth}{0.5pt} 
\renewcommand{\headrule}{\hbox to\headwidth{\color{umuBlue}\leaders\hrule height \headrulewidth\hfill}}

\usepackage{titlesec}

% Formato de capítulos
\titleformat{\chapter}[display]
  {\normalfont\sffamily\huge\bfseries\color{umuBlue}}
  {\filleft\fontsize{60}{60}\selectfont\color{grayText!30}\thechapter}
  {-1ex}
  {\filleft\titlerule[2pt]\vspace{1ex}}
  [\vspace{-1ex}]

% Formato de secciones
\titleformat{\section}
  {\normalfont\sffamily\Large\bfseries\color{umuBlue}}
  {\thesection}{1em}{}

% Formato de subsecciones
\titleformat{\subsection}
  {\normalfont\sffamily\large\bfseries\color{umuBlue}}
  {\thesubsection}{1em}{}

% ==========================================================
% DATOS DEL DOCUMENTO
% ==========================================================
\newcommand{\tituloTFG}{Aplicación Web Responsiva para la Gestión Empresarial Multi-Sede}
\newcommand{\subtituloTFG}{Diseño e Implementación de una Plataforma de Gestión de Recursos Humanos y Operaciones con Control de Roles}

% DATOS PERSONALES
\newcommand{\autor}{José Luis García Valverde}
\newcommand{\dni}{49309752Y}
\newcommand{\email}{jl.garciavalverde@um.es}

% DATOS ACADÉMICOS
\newcommand{\director}{Jose Antonio Miñarro Gimenez}
\newcommand{\grado}{Grado en Ingeniería Informática} 
\newcommand{\universidad}{Universidad de Murcia}
\newcommand{\facultad}{Facultad de Informática}
\newcommand{\fecha}{Diciembre de 2025}

% ==========================================================
% INICIO DEL DOCUMENTO
% ==========================================================
\begin{document}

% --- PORTADA ---
\begin{titlepage}
    \centering
    \thispagestyle{empty} 
    
    % Logos
    \begin{minipage}{0.45\textwidth}
        \flushleft
        \includegraphics[height=2cm]{imagenes/logo_um.jpg} 
    \end{minipage}
    \hfill
    \begin{minipage}{0.45\textwidth}
        \flushright
        \includegraphics[height=2cm]{imagenes/logo_fium.png} 
    \end{minipage}
    
    \vspace{2cm}
    
    {\scshape\LARGE \universidad \par}
    \vspace{0.2cm}
    {\scshape\Large \facultad \par}
    \vspace{0.5cm}
    {\scshape\large \grado \par}
    
    \vspace{2cm}
    
    % Título
    \noindent\rule{\linewidth}{2pt} \par
    \vspace{0.5cm}
    {\huge\bfseries\sffamily\color{umuBlue} \tituloTFG \par}
    \vspace{0.8cm}
    {\Large \itshape \subtituloTFG \par}
    \vspace{0.5cm}
    \noindent\rule{\linewidth}{2pt} \par
    
    \vfill
    
    % Bloque de Autor y Tutor
    \begin{minipage}{0.9\textwidth}
        \begin{flushright}
            \large
            \textbf{Alumno:} \autor \\
            \small DNI: \dni \\
            \small \email \\
            \vspace{0.8cm}
            \large
            \textbf{Tutor:} \director
        \end{flushright}
    \end{minipage}
    
    \vfill
    {\large \fecha \par}
\end{titlepage}

% --- PRELIMINARES ---
\pagenumbering{roman} 
\setcounter{page}{1}

% Resumen / Abstract
\chapter*{Resumen}
\addcontentsline{toc}{chapter}{Resumen}
Este Trabajo de Fin de Grado presenta el diseño y desarrollo de una plataforma web integral para la gestión empresarial en entornos multi-sede. La solución aborda la problemática de la fragmentación de datos en organizaciones distribuidas, centralizando la gestión de Recursos Humanos, operaciones y control de inventario en un único sistema.

Desarrollada bajo el stack tecnológico PERN (PostgreSQL, Express, React, Node.js) y reforzada con TypeScript, la aplicación ofrece una experiencia de usuario moderna y adaptativa (Mobile-First). El sistema implementa un robusto control de acceso basado en roles (RBAC) que diferencia claramente las funciones estratégicas (CEO), tácticas (Gerentes) y operativas (Empleados), optimizando así los flujos de trabajo y la toma de decisiones basada en datos.

\vspace{1cm}
\textbf{Palabras clave:} ERP, Gestión Multi-sede, Desarrollo Web, React, Node.js, SaaS.

\newpage
\tableofcontents
\newpage
\listoffigures
\newpage
\listoftables
\newpage

% --- CONTENIDO PRINCIPAL ---
\pagenumbering{arabic} 
\setcounter{page}{1}

% ==========================================================
% CAPÍTULOS
% ==========================================================

\chapter{Introducción}
\label{ch:introduccion}

\section{Contexto y Motivación}

La transformación digital ha dejado de ser una ventaja competitiva para convertirse en un requisito de supervivencia en el panorama empresarial actual. Las organizaciones con estructuras distribuidas o multi-sede enfrentan un desafío particular: la necesidad de una \textbf{centralización analítica} que proporcione a la dirección visibilidad global sobre el negocio, sin sacrificar la \textbf{agilidad operacional} que demandan las sedes locales para responder a su entorno específico.

\subsection*{La Problemática de los Silos de Información}

En la práctica, muchas PYMEs y empresas medianas del sector retail y servicios (cadenas de tiendas, restaurantes, centros de fitness, establecimientos de logística) adolecen de una fragmentación tecnológica conocida como ``silos de información''. Los datos críticos—Recursos Humanos, gestión de inventario, rendimiento operativo—residen dispersos en sistemas desconectados, software legado o, frecuentemente, en hojas de cálculo no estandarizadas y propensas a errores humanos.

Esta desconexión se traduce en tres problemas operativos concretos:

\begin{itemize}
    \item \textbf{Ceguera estratégica en la dirección:} Los directivos carecen de un cuadro de mando integrado que muestre en tiempo real métricas consolidadas (ventas globales, ocupación de plantilla, stock disponible) en cada sede. Sin esta visibilidad, la toma de decisiones se ralentiza y la detección de anomalías es reactiva, no proactiva. Esto puede retrasar decisiones críticas de expansión, cierre de sedes o reallocation de recursos en varias semanas.
    
    \item \textbf{Sobrecarga administrativa en gerentes locales:} Los responsables de cada sede se ven obligados a gestionar manualmente la asignación de turnos, reconciliar hojas de presencia con sistemas de nómina desacoplados y mantener inventarios en spreadsheets. Estudios del sector indican que los gerentes dedican entre 4 y 8 horas semanales a tareas administrativas no estratégicas, reduciendo el tiempo disponible para supervisión, formación de personal y mejora operativa.
    
    \item \textbf{Experiencia deficiente del empleado:} Los trabajadores operativos, típicamente sin acceso a ordenador de escritorio, no pueden consultar sus turnos próximos, solicitar permutas, o acceder a documentación laboral desde sus dispositivos personales. Esta fricción reduce satisfacción, incrementa consultas innecesarias al gerente y contribuye a una mayor rotación de personal.
\end{itemize}

\subsection*{Solución Propuesta}

El presente Trabajo de Fin de Grado ahonda en la necesidad de resolver estos retos mediante el diseño e implementación de una \textbf{plataforma web integral y moderna}. El proyecto propone un \textbf{ERP modular} que unifique la gestión operativa con el análisis de negocio en un único ecosistema digital, escalable y accesible desde cualquier dispositivo. De esta forma, se eliminan las barreras tradicionales entre sede central y sucursales, permitiendo que información fluya en tiempo real y que cada rol (directivo, gerente, empleado) disponga exactamente de lo que necesita para desempeñar su labor de forma eficiente.

\section{Objetivos del Proyecto}

\subsection{Objetivo General}
Diseñar, desarrollar y desplegar una aplicación web de gestión empresarial (ERP modular) basada en una arquitectura Cliente-Servidor moderna, que centralice la administración de recursos humanos y operaciones en entornos multi-sede, garantizando la consistencia de datos, la seguridad en el acceso y una experiencia de usuario adaptativa a diferentes dispositivos.

\subsection{Objetivos Específicos}
Para alcanzar el objetivo general, se establecen las siguientes metas organizadas en tres categorías: técnicas, funcionales y de calidad.

\subsubsection{Objetivos Técnicos}
\begin{itemize}
    \item \textbf{Diseño de Arquitectura Escalable y Desacoplada:} Implementar una arquitectura Cliente-Servidor mediante una API RESTful (Backend en Node.js/Express) e interfaz SPA (Frontend en React), permitiendo:
    \begin{itemize}
        \item Separación clara de responsabilidades que facilita el desarrollo y mantenimiento de una sola persona.
        \item Independencia de componentes para iterar y testear cada parte de forma aislada.
        \item Preparación para escalar el proyecto con nuevos clientes (aplicación móvil nativa, dashboards de terceros) sin refactorizar la lógica de negocio central.
        \item Mayor legibilidad y simplificación del debugging al aislar problemas por capa (frontend vs. backend).
    \end{itemize}
    
    \item \textbf{Seguridad Integral y Control de Acceso Basado en Roles (RBAC):} Desarrollar un módulo de autenticación y autorización robusto que garantice:
    \begin{itemize}
        \item Identificación segura mediante JWT con expiración temporal.
        \item Autorización granular basada en tres roles jerárquicos (CEO, Gerente, Empleado) siguiendo el principio de mínimo privilegio.
        \item Auditoría de acceso a datos sensibles (nóminas, información contractual).
    \end{itemize}
\end{itemize}

\subsubsection{Objetivos Funcionales}
\begin{itemize}
    \item \textbf{Dashboard Ejecutivo Integral:} Implementar un cuadro de mando para directivos que proporcione:
    \begin{itemize}
        \item Visualización comparativa de beneficios y ventas con granularidad por sede y período temporal.
        \item Métricas de RRHH consolidadas (headcount, tasas de absentismo, desviaciones de plantilla).
        \item Acceso centralizado a comentarios constructivos de empleados filtrados por ubicación.
        \item \textbf{Métrica de éxito:} Reducción de tiempo para obtener un informe de desempeño multi-sede de 2-3 horas a menos de 5 minutos.
    \end{itemize}
    
    \item \textbf{Planificador de Turnos y Gestión de Incidencias:} Reducir carga administrativa del gerente mediante:
    \begin{itemize}
        \item Herramienta visual para asignación de turnos con detección de conflictos de solapamiento.
        \item Centro de notificaciones para solicitudes de empleados (cambios de turno, permisos).
        \item Gestión de permutas entre empleados con aprobación/rechazo del gerente.
        \item \textbf{Métrica de éxito:} Reducción de tiempo de asignación semanal de turnos de 2-3 horas a 30 minutos.
    \end{itemize}
    
    \item \textbf{Control de Inventario Localizado y CRUD de Productos:} Gestión desacoplada por sede que incluya:
    \begin{itemize}
        \item Catálogo de productos con precio, descripción y disponibilidad.
        \item Herramientas de ajuste de stock y auditoría de movimientos.
        \item Generación de alertas por bajo stock.
    \end{itemize}
    
    \item \textbf{Portal Auto-Gestión para Empleados:} Proporcionar al personal operativo acceso móvil a:
    \begin{itemize}
        \item Visualización de próximos turnos asignados (semanal y mensual).
        \item Solicitud formalizada de cambios de turno y visualización de estado de solicitudes.
        \item Descarga de nóminas y documentación laboral.
        \item \textbf{Métrica de éxito:} Reducción de consultas administrativas en un 40\% en los primeros 3 meses post-lanzamiento.
    \end{itemize}
\end{itemize}

\subsubsection{Objetivos de Calidad y Testing}
\begin{itemize}
    \item \textbf{Cobertura de Testing Completa:} Implementar estrategia de testing multinivel:
    \begin{itemize}
        \item Unit tests para lógica de negocio crítica (cálculo de horas, validación de turnos).
        \item Integration tests para flujos de datos entre frontend-backend.
        \item E2E tests para casos de uso críticos (login, asignación de turnos, descarga de nómina).
        \item \textbf{Métrica de éxito:} Cobertura de código mínima del 80\% en módulos críticos.
    \end{itemize}
    
    \item \textbf{Accesibilidad Universal (Responsive Design):} Garantizar usabilidad en todos los dispositivos:
    \begin{itemize}
        \item Interfaz adaptativa que funciona sin degradación en desktop, tablet y móvil (viewport desde 320px).
        \item Tiempo de carga inicial $\leq$ 3 segundos en conexión 3G.
        \item Compatibilidad con navegadores modernos (Chrome, Firefox, Safari, Edge versiones recientes).
    \end{itemize}
\end{itemize}

\section{Estructura de la Memoria}
El presente documento se organiza en los siguientes capítulos:

\begin{itemize}
    \item \textbf{Capítulo \ref{ch:dominio}: Dominio y Estado del Arte.} Analiza el contexto de los sistemas ERP actuales, identifica los tres perfiles de usuario (CEO, Gerente, Empleado) y realiza un estudio comparativo de las soluciones existentes en el mercado (ERPs monolíticos, plataformas RRHH, herramientas especializadas de turnos) para justificar la propuesta de valor diferencial del proyecto.
    
    \item \textbf{Capítulo \ref{ch:tecnologia}: Selección Tecnológica y Arquitectura.} Detalla la arquitectura de software diseñada (Cliente-Servidor) mediante diagrama conceptual, justifica la elección del stack tecnológico (PERN + TypeScript) con argumentación técnica de beneficios para escalabilidad, mantenibilidad y seguridad, y explica las decisiones sobre herramientas de soporte (git, testing frameworks, etc.).
    
    \item \textbf{Capítulo \ref{ch:funcionalidad}: Análisis Funcional y Subsistemas.} Desglosa en detalle las funcionalidades del sistema organizadas por perfiles de usuario, describiendo los tres subsistemas principales (Dirección, Gestión Local, Operativo), sus casos de uso críticos, pantallas asociadas y flujos de usuario, con énfasis en la experiencia responsiva.
\end{itemize}
\chapter{Dominio y Estado del Arte}
\label{ch:dominio}

\section{Dominio de la Aplicación}

El sistema desarrollado se enmarca en el dominio de los \textbf{Sistemas de Planificación de Recursos Empresariales (ERP)} de nueva generación. Específicamente, el proyecto se posiciona como una solución \textbf{SaaS (Software as a Service)} vertical, especializada en la gestión integral de \textbf{Recursos Humanos (RRHH)} y la orquestación de \textbf{Operaciones Multi-Sede} para el sector retail y servicios.

\subsection*{Contexto de Mercado}

El mercado de soluciones ERP está en un momento de transición significativa. Aunque los grandes sistemas monolíticos (SAP, Oracle) mantienen presencia en grandes corporaciones, existe una tendencia emergente hacia soluciones SaaS modulares que democratizan el acceso a la tecnología empresarial. En 2026, el mercado de software de gestión para PYMEs evidencia demanda creciente por:

\begin{itemize}
    \item \textbf{Modularidad y flexibilidad:} Las empresas medianas requieren soluciones que se adapten a su operativa, no al revés.
    \item \textbf{Experiencia móvil nativa:} Con más del 60\% de la fuerza laboral en retail sin puesto fijo, el acceso móvil es ya requisito fundamental.
    \item \textbf{Integración seamless:} Conectar con sistemas legacy sin implementaciones costosas.
    \item \textbf{Costos predecibles:} Modelo SaaS con precios mensuales, no licencias perpetuas de seis cifras.
\end{itemize}

\subsection{Roles y Flujos de Usuario}

La arquitectura de información y la lógica de negocio del sistema se articulan en torno a tres perfiles de usuario jerarquizados, diseñados para cubrir las necesidades estratégicas, tácticas y operativas de la organización:

\begin{enumerate}
    \item \textbf{CEO (Nivel Estratégico):} Requiere una visión holística del negocio. Su interacción se centra en cuadros de mando (\textit{dashboards}) agregados que permiten el análisis comparativo de rendimiento entre sedes y el control macroeconómico de la estructura organizacional. \textit{Caso de uso:} El lunes a las 9:00 AM, el CEO accede desde su móvil a un resumen ejecutivo: ventas consolidadas del fin de semana (+12\% respecto a la semana anterior), ocupación de plantilla (95\%), y alertas de inventario crítico. En segundos, identifica que Madrid con 89\% de ocupación requiere atención.
    
    \item \textbf{Gerente de Sede (Nivel Táctico-Operativo):} Es el responsable de la ejecución eficiente en su ubicación específica. Su flujo de trabajo prioriza la agilidad administrativa en la gestión de turnos, resolución de incidencias y control de inventario local. \textit{Caso de uso:} El martes, una llamada de baja de último minuto afecta al turno de tarde. El gerente abre la aplicación desde el mostrador, visualiza empleados disponibles en el calendario, asigna un sustituto en 90 segundos, genera automáticamente la notificación al empleado, y el sistema ajusta el cálculo de horas extra si procede.
    
    \item \textbf{Empleado (Nivel Operativo):} Constituye el grueso de la fuerza laboral. Utiliza la plataforma principalmente en dispositivos móviles como portal de autogestión. \textit{Caso de uso:} Una vendedora necesita cambiar su turno del próximo jueves. Desde su móvil, visualiza qué compañeros están disponibles, selecciona una permuta, solicita aprobación, y en 2 horas recibe confirmación del gerente. También descarga su nómina mensual desde el mismo app.
\end{enumerate}

% TODO: DIAGRAMA DE ROLES Y NIVELES
% Descomentar cuando tengas el diagrama
% \begin{figure}[h]
%     \centering
%     \includegraphics[width=0.6\textwidth]{imagenes/roles_piramide.png}
%     \caption{Pirámide de roles: Nivel estratégico (CEO), táctico (Gerente), operativo (Empleado). Cada nivel requiere diferente granularidad de datos.}
%     \label{fig:roles_pyramid}
% \end{figure}
%
% NOTA: Esta imagen podría ser un diagrama Mermaid o dibujado en PowerPoint mostrando:
% - Triángulo con CEO arriba, Gerente a medio, Empleados abajo
% - Flechas de comunicación vertical (feedback)
% - % de tiempo de uso de la plataforma
% ========================================================

\section{Estado del Arte y Análisis de la Competencia}
El estudio del estado del arte examina las soluciones de gestión empresarial vigentes en el mercado, contrastando los grandes sistemas ERP monolíticos tradicionales frente a las soluciones verticales modernas y plataformas de nicho.

\subsection{Análisis de Soluciones Existentes (2026)}

Para identificar oportunidades de mejora, se ha realizado un análisis comparativo de diversas herramientas vigentes en el mercado actual, categorizadas según su aproximación al problema. La Tabla \ref{tab:comparativa} resume este estudio considerando coste, experiencia móvil y limitaciones funcionales detectadas.

\begin{table}[h]
    \centering
    \caption{Comparativa de soluciones existentes: análisis de brechas y costos aproximados (2026)}
    \label{tab:comparativa}
    \vspace{0.3cm}
    \begin{tabularx}{\textwidth}{l X c X}
        \toprule
        \textbf{Solución} & \textbf{Coste/año} & \textbf{Móvil} & \textbf{Limitaciones fundamentales} \\
        \midrule
        \textbf{SAP B1, Dynamics 365} & \$40K-80K & NO & Prohibitivos para PYMEs. Rigidez estructural. Meses de curva de aprendizaje. No diseñados para retail multi-sede. \\
        \addlinespace
        \textbf{Factorial, Workday, Bizneo} & \$1-3/emp/mes & SÍ & Excelentes en RRHH administrativo, pero inventario y turnos complejos son secundarios. Gestión multi-sede débil. \\
        \addlinespace
        \textbf{Sesame HR, Combo, HRLOG} & \$500-2K/mes & DÉBIL & Especializados en turnos pero generan silos de información. Datos operativos no se integran con KPIs de negocio. Sin dashboard de dirección. \\
        \addlinespace
        \textbf{Square, Toast (POS+RRHH)} & \$300-1.5K/mes & BUENO & Fuertes en POS y ventas, débiles en planificación de RRHH multi-sede y gestión centralizada. \\
        \addlinespace
        \textbf{\textbf{Solución Propuesta (TFG)}} & \textbf{SaaS flexible} & \textbf{SÍ+} & \textbf{Integra todo lo anterior: RRHH + Operativa + Dashboard CEO. Móvil nativo. Bajo coste. Flexible.} \\
        \bottomrule
    \end{tabularx}
\end{table}

% ==========================================================
% TODO: IMÁGENES DEL ESTADO DEL ARTE
% ==========================================================
% Descomentar y completar cuando tengas acceso a screenshots
% 
% Imagen 1: Dashboard de Workday (RRHH estándar)
% \begin{figure}[h]
%     \centering
%     \includegraphics[width=0.85\textwidth]{imagenes/workday_dashboard.png}
%     \caption{Interfaz típica de Workday: gestión RRHH centralizada pero compleja para operativa multi-sede.}
%     \label{fig:workday_example}
% \end{figure}
% 
% Imagen 2: Calendario de turnos Sesame HR (especializado pero aislado)
% \begin{figure}[h]
%     \centering
%     \includegraphics[width=0.85\textwidth]{imagenes/sesame_turnos.png}
%     \caption{Interfaz de Sesame HR: excelente para gestión de turnos, pero sin integración con KPIs de negocio o visión multi-sede.}
%     \label{fig:sesame_example}
% \end{figure}
% 
% Imagen 3: Dashboard retail Square/Toast (híbrido POS+RRHH)
% \begin{figure}[h]
%     \centering
%     \includegraphics[width=0.85\textwidth]{imagenes/square_dashboard.png}
%     \caption{Interfaz de Square: fuerte en POS y ventas, pero débil en planificación de RRHH centralizada.}
%     \label{fig:square_example}
% \end{figure}
%
% FUENTES SUGERIDAS PARA OBTENER SCREENSHOTS:
% - Workday: https://www.workday.com/en-US/products/resource-management/demo.html
% - Sesame HR: https://www.sesame.com/en/screenshots (sección demo)
% - Square for Restaurants/Toast: https://squareup.com/en-us/point-of-sale/restaurant (galería de imágenes)
% ==========================================================

\subsection{Propuesta de Valor Única (PVU)

Como respuesta a las carencias identificadas en el mercado, el sistema propuesto define su propuesta de valor en la \textbf{democratización de la gestión multi-sede eficiente} mediante una plataforma integrada diseñada específicamente para retail y servicios. Los pilares diferenciadores son:

\begin{enumerate}
    \item \textbf{Segregación Contextual por Rol (``Right-Sizing Complexity''):} A diferencia de los ERPs genéricos que abruman con opciones, la interfaz adapta dinámicamente lo que cada usuario ve. El CEO ve KPIs y alertas estratégicas; el gerente ve tablas de turnos y stock; el empleado ve su próximo turno y nómina. \textit{Ejemplo:} Un CEO accede a la misma plataforma que sus 200 empleados, pero ve un 5\% de las funcionalidades disponibles—exactamente lo que necesita. Esto mejora significativamente la adopción respecto a sistemas genéricos.
    
    \item \textbf{Accesibilidad Ubicua (Mobile-First, Verdaderamente):} A diferencia de otras plataformas con ``responsive design'' que simulan móvil, esta es nativa mobile-first desde su arquitectura. Funciona perfectamente en conexiones lentas (3G), sin dependencia de JavaScript pesado, con sincronización automática cuando se recupera la conectividad. \textit{Ejemplo:} Un empleado en la calle con 2 barras de señal puede solicitar una permuta, descargar su nómina, y ver cambios de turno sin esperar.
    
    \item \textbf{Unificación de Datos en Tiempo Real:} A diferencia de herramientas especializadas que generan silos, todos los datos (turnos, ventas, inventario, RRHH) convergen en una única base de datos. Cuando un turno se modifica, el cálculo de horas actualiza dinámicamente el coste de nómina estimado; cuando un empleado falta, el dashboard del CEO lo refleja al instante en las métricas de rendimiento. \textit{Ejemplo:} No existen hojas de cálculo desactualizadas—el sistema es la ``fuente única de verdad''.
    
    \item \textbf{Comunicación Vertical Integrada:} Implementa un canal de feedback estructurado que conecta operativamente la base (``pie de tienda'') con la dirección. Un empleado reporta un problema de stock o clima laboral, el gerente lo clasifica, y el CEO lo ve en un tablero de incidencias. Elimina intermediarios y agiliza decisiones.
\end{enumerate}

\subsection*{Diferencial Tecnológico}

Desde una perspectiva técnica, la solución aprovecha las tendencias tecnológicas de 2026:
\begin{itemize}
    \item \textbf{Stack moderno open-source:} Utiliza PERN + TypeScript, tecnologías ampliamente adoptadas con gran comunidad. Esto asegura que el código sea mantenible, escalable y sin ``vendor lock-in''.
    \item \textbf{API RESTful de primera clase:} Diseñada para permitir integraciones futuras (POS, ERP legacy, BI tools) sin refactorización.
    \item \textbf{Seguridad granular (RBAC):} Control de acceso basado en roles con auditoría, cumpliendo regulaciones de datos sensibles (nóminas).
\end{itemize}
% ==========================================================
% CAPÍTULO 3: SELECCIÓN TECNOLÓGICA
% ==========================================================
\chapter{Selección Tecnológica y Arquitectura}
\label{ch:tecnologia}

\section{Arquitectura del Sistema}

Para satisfacer los requisitos no funcionales de interactividad, inmediatez y escalabilidad, se ha diseñado una arquitectura basada en \textbf{Componentes} y fuertemente desacoplada. La solución sigue el patrón de separación de responsabilidades entre el \textit{Frontend} (Cliente) y el \textit{Backend} (Servidor), estableciendo la comunicación a través de una API RESTful estandarizada.

Esta separación arquitectónica ofrece múltiples ventajas:

\begin{itemize}
    \item \textbf{Claridad de responsabilidades:} Frontend gestiona UI/UX y estado local; Backend maneja lógica de negocio y persistencia.
    \item \textbf{Mantenibilidad mejorada:} Cambios en una capa no requieren refactorización de la otra, facilitando el desarrollo iterativo individual.
    \item \textbf{Testabilidad:} Cada capa puede ser testeada independientemente (unitarios, integración, E2E).
    \item \textbf{Escalabilidad futura:} Permite integración de nuevos clientes (aplicación móvil nativa, dashboards BI, integraciones POS) sin modificar la lógica de negocio central.
\end{itemize}

\subsection*{Flujo de Datos}

El flujo de comunicación entre capas es bidireccional y asíncrono:

\begin{enumerate}
    \item \textbf{Frontend (React):} Usuario interactúa con la UI. JavaScript captura eventos, valida inputs localmente con Zod, y prepara peticiones HTTP.
    \item \textbf{API REST (Express):} Recibe petición, valida esquema de nuevo con Zod, ejecuta lógica de negocio, consulta/actualiza datos en PostgreSQL vía Prisma.
    \item \textbf{Base de Datos (PostgreSQL):} Persiste datos con garantías ACID. Prisma genera SQL optimizado automáticamente.
    \item \textbf{Respuesta al Cliente:} JSON tipado se serializa y transmite al frontend. TanStack Query gestiona caché inteligentemente.
\end{enumerate}

\begin{figure}[h]
    \centering
    % Asegúrate de actualizar la imagen si el diagrama cambia, aunque la arquitectura base es la misma.
    \includegraphics[width=0.8\textwidth]{imagenes/React Client Architecture-2025-12-09-180737.png}
    \caption{Diagrama de alto nivel de la arquitectura Cliente-Servidor propuesta.}
    \label{fig:arquitectura}
\end{figure}

\section{Stack Tecnológico (PERN + TypeScript)}

La pila tecnológica seleccionada para la implementación se conoce en la industria como \textbf{PERN Stack} (PostgreSQL, Express, React, Node.js). Adicionalmente, se ha incorporado \textbf{TypeScript} de manera transversal en todo el proyecto para garantizar la seguridad de tipos y robustez del código.

A continuación, se detalla y justifica la elección de cada tecnología, contrastándola con alternativas descartadas.

\subsection{Frontend: Interfaz de Usuario}
El desarrollo de la interfaz de usuario prioriza la experiencia ``Mobile-First'' y la reactividad, requisitos esenciales para que empleados en diferentes sedes y roles (CEO, gerentes, empleados) puedan acceder desde múltiples dispositivos.

\begin{itemize}
    \item \textbf{React.js:} Biblioteca de JavaScript para la construcción de interfaces.
    \\ \textit{Justificación Comercial:} La reutilización de componentes es crítica para un equipo de desarrollo en solitario. Elementos como el ``Calendario de Turnos'', ``Panel de Notificaciones'' y ``Formularios de Turnos'' se utilizan repetidamente en diferentes contextos. Este enfoque reduce bugs y acelera time-to-market. Además, el éxito comprobado de React en aplicaciones empresariales como Facebook, Netflix e Instagram garantiza madurez y comunidad.
    \\ \textit{Justificación Técnica:} El modelo de componentes funcionales con hooks permite lógica clara y testeable. La arquitectura unidireccional de flujo de datos (props down, events up) previene bugs de estado común en aplicaciones complejas.

    \item \textbf{Vite:} Entorno de desarrollo y herramienta de construcción (bundler).
    \\ \textit{Justificación Comercial:} El ciclo de desarrollo es crítico en el Scrum. Vite reduce tiempos de recarga a <100ms vs 1-2s en Webpack, permitiendo feedback inmediato. Esto traduce a 2-3 horas productivas adicionales por semana de desarrollo, significativo en un proyecto de 6 sprints.
    \\ \textit{Justificación Técnica:} Uso de módulos ES6 nativos durante desarrollo elimina step de bundling, vs Webpack que requería transpilación. En producción, Vite genera bundles 30-40\% más pequeños con tree-shaking automático.
    
    \item \textbf{TanStack Query (React Query):} Gestor de estado asíncrono.
    \\ \textit{Justificación de Negocio:} La gestión manual de caché y state es la principal fuente de bugs frontend (datos desincronizados, pérdida de cambios, UX pobre). TanStack Query encapsula 80\% de la lógica asíncrona común, reduciendo bugs en desarrollo de features y acelerando entregas.
    \\ \textit{Justificación Técnica:} Mantiene dos fuentes de verdad sincronizadas automáticamente: estado local (UI) y estado servidor (DB). Background refetching detecta cambios en tiempo real; retry automático mejora resiliencia en redes débiles (crucial para empleados en tienda).

    \item \textbf{Tailwind CSS:} Framework de utilidades CSS.
    \\ \textit{Justificación de Negocio:} La consistencia visual entre sedes es importante para brand della PYME. Tailwind impone paleta de colores, espaciado y tipografía uniformes mediante configuración central, vs CSS custom que es propenso a inconsistencias. Reducción de time-to-design permite enfoque en lógica de negocio.
    \\ \textit{Justificación Técnica:} El enfoque ``utility-first'' genera CSS mucho más pequeño (purga automática de CSS no usado) que librerías CSS pesadas. Bundle CSS final: ~20kB vs 50+ con Bootstrap.

    \item \textbf{Day.js:} Librería de manipulación de fechas.
    \\ \textit{Justificación de Negocio:} Errores de fecha/hora frecuentemente causan reclamaciones de empleados (paga incorrecta). Una librería robusta con soporte de timezones previene bugs costosos. Day.js elegido por peso ultra-bajo (2kB vs Moment.js 67kB).
    \\ \textit{Justificación Técnica:} Inmutabilidad por diseño (cada operación retorna nuevo objeto) previene errores de mutación silenciosa comunes en JavaScript vanilla.
\end{itemize}

\subsection{Backend: Lógica de Negocio y API}
El servidor actúa como orquestador de la lógica de negocio, seguridad y acceso a datos. La selección de tecnologías backend prioriza confiabilidad (datos de nómina/contratos no pueden perderse), velocidad de iteración (desarrollador único), y escalabilidad futura.

\begin{itemize}
    \item \textbf{Node.js y Express.js:} Entorno de ejecución y framework web minimalista.
    \\ \textit{Justificación de Negocio:} Ecosistema JavaScript unificado frontend-backend permite que un único desarrollador sea productivo sin context-switching entre lenguajes. Express es el framework más usado en industria (comunidad, librerías, ejemplos), minimizando riesgo técnico.
    \\ \textit{Justificación Técnica:} El modelo de E/S no-bloqueante (event-driven) de Node.js es optimal para sistemas REST intensivos en I/O: cada petición es una operación asíncrona que no bloquea otras. Pueden manejarse miles de conexiones concurrentes con bajo overhead de memoria. Comparado con modelo thread-per-request de Java/C\#, Node.js requiere ~50MB/conexión vs ~1MB.

    \item \textbf{Zod:} Librería de validación de esquemas (Schema Validation).
    \\ \textit{Justificación de Negocio:} Errores de validación en transacciones económicas son críticos. Zod impone contract strict entre frontend y backend, detectando discrepancias en datos (ej: una nómina con campo de sueldo como string en lugar de number). El costo de un bug de validación desapercibido (sobrepagas, sueldos truncados) es exponencialmente mayor que invertir en validación exhaustiva.
    \\ \textit{Justificación Técnica:} Zod permite inferir tipos TypeScript directamente del esquema. Si cambio el schema, TypeScript me obliga a actualizar el código que consume (compiler catches breaking changes). Sin esto, cambios a la API pueden romper clientes sin saberlo hasta runtime.

    \item \textbf{JWT (JSON Web Tokens):} Estándar de autenticación.
    \\ \textit{Justificación de Negocio:} Autenticación stateless es requisito para escalar. Si guardo sesiones en memoria en el servidor, escalabilidad horizontal requeriría sincronización entre instancias (sticky sessions, shared cache). JWT elimina ese problema: cada petición es auto-suficiente. Perfectamente escalable. Además, JWT permite pre-autenticar via QR codes, RFID, biometría sin cambiar la API.
    \\ \textit{Justificación Técnica:} JWT contiene claims (UserID, rol) firmados criptográficamente. Servidor no necesita conexión a DB para validar; solo verifica firma. Reduce latencia de autenticación a <1ms vs 10-50ms de búsqueda en DB.
\end{itemize}

\subsection{Persistencia de Datos}
\begin{itemize}
    \item \textbf{PostgreSQL:} Sistema de gestión de bases de datos relacional (RDBMS).
    \\ \textit{Justificación de Negocio:} El sistema maneja datos financieros críticos: nóminas, contratos, distribución de ganancias por sede. La pérdida o corrupción de datos es inaceptable. PostgreSQL proporciona ACID (Atomicity, Consistency, Isolation, Durability), garantizando integridad incluso ante crashes. Además, soporte para foreign keys y constraints aseguran que la base de datos "rechace" operaciones inválidas (ej: no cree turno para empleado que no existe). También es open-source, evitando lock-in comercial.
    \\ \textit{Justificación Técnica:} PostgreSQL soporta transacciones complejas (ej: crear turno + decrementar cuota de horas en una operación atómica). Incluye índices sofisticados, JSON nativo, y Full-Text Search. Probado en miles de sistemas empresariales, de Twitter a Instagram.

    \item \textbf{Prisma ORM:} Herramienta de Mapeo Objeto-Relacional.
    \\ \textit{Justificación de Negocio:} SQL inyección es una de las vulnerabilidades más comunes y costosas. Prisma previene esto automáticamente via parameterized queries. También proporciona migraciones versionadas, permitiendo auditar cada cambio a la estructura de datos (crítico para compliance y debugging). En desarrollo, ``prisma studio'' permite visualizar datos sin escribir SQL, acelerando debugging.
    \\ \textit{Justificación Técnica:} Prisma genera TypeScript types automáticamente del schema. Si cambio la tabla User en PostgreSQL, ejecutar ``npx prisma generate'' actualiza tipos TypeScript. TypeScript compiler obliga a todos los \.ts files que usan User a adaptarse. Bugs de schema dormant son imposibles. Comparado con escritura manual de tipos o ORMs genéricos, reduce bugs de incompatibilidad 80\%.
\end{itemize}

\subsection{Herramientas de Soporte}

\begin{itemize}
    \item \textbf{Git y GitHub:} Control de versiones distribuido.
    \\ \textit{Justificación de Negocio:} Permite trazabilidad completa de cambios. Si ocurre un bug en producción (ej: empleados pagan 10\% menos de lo debido), puedo revisar exactamente qué código cambió en los últimos commits y revertir si es necesario. Además, GitHub Actions permite CI/CD: cada commit corre tests automáticamente, detectando bugs antes de que lleguen a producción.
    \\ \textit{Justificación Técnica:} Modelo distribuido permite que múltiples desarrolladores trabajen sin conflictos. Rama \texttt{main} protegida requiere que todo cambio pase tests y revisión antes de mergear (previene merges accidentales de código roto). Squash commits mantienen historia limpia pese a múltiples iteraciones locales.
    
    \item \textbf{ESLint y Prettier:} Linting y code formatting.
    \\ \textit{Justificación de Negocio:} Por cada issue de formato (inconsistencia de espacios, semicolons) que causa conflicto de merge, se pierde 15-30 minutos. Prettier elimina esto: formatea automáticamente en cada save, precommit-hooks evitan commits mal formateados. Tiempo ahorrado se invierte en features.
    \\ \textit{Justificación Técnica:} ESLint (Con preset airbnb-typescript) identifica bugs comunes (variables no usadas, tipos implícitos any, errores lógicos). Pre-commit hooks ejecutan automáticamente antes de que código llegue a repositorio, asegurando que main branch nunca contiene código que viola estándares.
\end{itemize}

\section{Estrategia de Testing}

La calidad de código es crítica en sistemas que manejan datos empresariales sensibles. Se implementa una estrategia de testing en tres niveles:

\begin{itemize}
    \item \textbf{Unit Tests (Jest):} Testeado en backend para funciones puras: cálculo de horas, validación de turnos, lógica de permisos RBAC. Target: 80\% de cobertura en módulos críticos.
    
    \item \textbf{Integration Tests (Supertest + Jest):} Validación de flujos API completos: crear usuario → asignar rol → consultar datos desde cliente. Incluye mock de PostgreSQL con test containers.
    
    \item \textbf{E2E Tests (Playwright):} Casos de uso completos desde UI: login CEO → acceder dashboard → filtrar por sede. Tests en navegador real (Chrome, Firefox).
\end{itemize}

\textbf{Ejecución:} Scripts en \texttt{package.json}: \texttt{npm run test}, \texttt{npm run test:e2e}. Integración con CI/CD para rechazar PR si tests fallan.

\section{Seguridad y Control de Acceso}

La seguridad es multinivel y se refuerza en cada capa:

\begin{itemize}
    \item \textbf{Autenticación (JWT):} 
    \begin{itemize}
        \item Access Token válido 15 minutos (minimiza exposición si es robado).
        \item Refresh Token válido 7 días, almacenado en HTTP-only cookie (protegido de XSS).
        \item Backend valida firma JWT en cada petición; rechaza si expirado o inválido.
    \end{itemize}
    
    \item \textbf{Autorización (RBAC):}
    \begin{itemize}
        \item Tres roles: \texttt{ADMIN} (CEO), \texttt{MANAGER} (Gerente), \texttt{EMPLOYEE} (Empleado).
        \item Middleware Express valida permisos antes de ejecutar lógica: \texttt{(req, res, next) => checkRole(req.user.role, REQUIRED\_ROLES)}.
        \item Filtrado de datos en queries: empleado solo ve sus turnos y nómina, gerente solo su sede.
    \end{itemize}
    
    \item \textbf{Validación en Frontend y Backend:}
    \begin{itemize}
        \item Frontend valida con Zod para UX rápido.
        \item Backend valida de nuevo por seguridad (nunca confiar en cliente).
    \end{itemize}
    
    \item \textbf{Protección de Datos Sensibles:}
    \begin{itemize}
        \item Nóminas y contratos se transmiten por HTTPS (TLS 1.2+).
        \item Contraseñas hasheadas con bcrypt (10 rounds) en DB.
        \item SQL injection prevenida por Prisma (parameterized queries).
    \end{itemize}
\end{itemize}

\section{Performance y Escalabilidad}

Optimizaciones implementadas para garantizar responsividad incluso al crecer:

\begin{itemize}
    \item \textbf{Frontend:}
    \begin{itemize}
        \item Code splitting: cada ruta carga solo su JS (lazy loading con React.lazy).
        \item Caché inteligente con TanStack Query: evita fetches innecesarios.
        \item Compression gzip en dist (Vite).
        \item Imágenes optimizadas (webp, srcset responsivo).
    \end{itemize}
    
    \item \textbf{Backend:}
    \begin{itemize}
        \item Índices PostgreSQL en columnas frequently queried: usuario\_id, sede\_id, fecha.
        \item Paging en listados (no traer 10K turnos sin paginar).
        \item Compression HTTP (gzip) de respuestas JSON grandes.
        \item Connection pooling con Prisma para no agotarBD.
    \end{itemize}
    
    \item \textbf{Infraestructura:}
    \begin{itemize}
        \item Docker containers permiten scaling horizontal (múltiples instancias del backend).
        \item Load balancer (nginx o AWS ALB) distribuye peticiones.
    \end{itemize}
\end{itemize}

\section{Deployment y DevOps}

Configuración para producción:

\begin{itemize}
    \item \textbf{Containerización:} Dockerfile para backend (Node.js image alpine, <100MB). Frontend buildado a static files (HTML/CSS/JS), servido por nginx.
    
    \item \textbf{Base de Datos:} PostgreSQL en container separado (volumen persistente) o servicio managed (AWS RDS). Backups automáticos diarios.
    
    \item \textbf{Entornos:}
    \begin{itemize}
        \item \texttt{Development:} localhost:3000 (frontend), localhost:3001 (backend). Database local.
        \item \texttt{Production:} Docker Compose o Kubernetes. Variables de entorno para credenciales (no hardcodeadas).
    \end{itemize}
    
    \item \textbf{Monitoreo:} Logs centralizados (stdout/stderr capturados por Docker). Alertas si API no responde.
\end{itemize}

\section{Análisis Comparativo: Alternativas Descartadas}

La Tabla \ref{tab:tech_comparison} justifica por qué se seleccionó PERN+TS sobre otras opciones del mercado.

\begin{table}[h]
    \centering
    \caption{Comparativa de stacks tecnológicos: razones de selección PERN}
    \label{tab:tech_comparison}
    \vspace{0.3cm}
    \small
    \setlength{\tabcolsep}{4pt}
    \renewcommand{\arraystretch}{1.2}
    \begin{tabularx}{\textwidth}{>
        {\raggedright\arraybackslash}p{0.22\textwidth}
        >{\raggedright\arraybackslash}p{0.26\textwidth}
        X}
        \toprule
        \textbf{Stack} & \textbf{Ventajas} & \textbf{Limitaciones / Razones de rechazo} \\
        \midrule
        \textbf{PERN + TypeScript} & Community grande, open-source, flexible, mobile-first facil, ACID DB, testing sencillo & Ninguno para este proyecto. Seleccionado. \\
        \addlinespace
        \textbf{MEAN/MERN (MongoDB)} & Schema flexible, rapido para prototipo & MongoDB pierde ACID en collections distribuidas. Para datos criticos (nominas) se requiere integridad. \\
        \addlinespace
        \textbf{Next.js (SSR fullstack)} & Menos boilerplate, SSR de serie & Next.js mezcla backend/frontend. Dificulta testing aislado. Overkill para esta SPA. \\
        \addlinespace
        \textbf{.NET + Azure} & Enterprise-grade, MSSQL robusto & Licenciado y costoso. Overkill para PYME. Menos community para features innovadores. \\
        \addlinespace
        \textbf{Django + PostgreSQL} & DRF excelente, admin built-in & Python frontend debil (no hay equivalente a React). Habria que usar React + Django API, similar a PERN pero con menos integraciones. \\
        \addlinespace
        \textbf{Java Spring + React} & Spring Boot robusto, mature & Verboso, lento de desarrollar en solitario. Overhead de JVM. Mejor para teams grandes. \\
        \bottomrule
    \end{tabularx}
\end{table}

% ==========================================================
% CAPÍTULO 4: ESPECIFICACIÓN FUNCIONAL
% ==========================================================
\chapter{Análisis Funcional y Subsistemas}
\label{ch:funcionalidad}

\section{Modelo de Datos}

El núcleo del sistema se articula en torno a las siguientes entidades principales y sus relaciones:

\begin{itemize}
    \item \textbf{User:} Representa cada usuario del sistema. Campos: \texttt{id}, \texttt{email}, \texttt{role} (ADMIN/MANAGER/EMPLOYEE), \texttt{password\_hash}, \texttt{sede\_id} (referencia a sede asignada).
    
    \item \textbf{Sede:} Sucursal física de la organización. Campos: \texttt{id}, \texttt{nombre}, \texttt{ubicación}, \texttt{manager\_id} (FK User), \texttt{created\_at}.
    
    \item \textbf{Shift (Turno):} Período de trabajo asignado a un empleado. Campos: \texttt{id}, \texttt{employee\_id} (FK User), \texttt{sede\_id} (FK Sede), \texttt{start\_time}, \texttt{end\_time}, \texttt{status} (PENDING/CONFIRMED/CANCELLED), \texttt{date}.
    
    \item \textbf{ShiftRequest:} Solicitud de cambio de turno. Campos: \texttt{id}, \texttt{employee\_id}, \texttt{requested\_shift\_id}, \texttt{status} (PENDING/APPROVED/REJECTED), \texttt{requested\_at}, \texttt{response\_date}.
    
    \item \textbf{Product:} Artículo del catálogo de la sede. Campos: \texttt{id}, \texttt{sede\_id}, \texttt{nombre}, \texttt{precio}, \texttt{stock\_actual}, \texttt{stock\_minimo}.
    
    \item \textbf{Payroll:} Documento de nómina. Campos: \texttt{id}, \texttt{employee\_id}, \texttt{periodo\_mes}, \texttt{file\_url}, \texttt{uploaded\_at}.
    
    \item \textbf{Feedback:} Comentarios de empleados. Campos: \texttt{id}, \texttt{employee\_id}, \texttt{content}, \texttt{category} (BUG/SUGGESTION/COMPLAINT), \texttt{status} (OPEN/IN\_REVIEW/RESOLVED), \texttt{created\_at}.
\end{itemize}

% TODO: DIAGRAMA ENTIDAD-RELACIÓN
% \begin{figure}[h]
%     \centering
%     \includegraphics[width=0.9\textwidth]{imagenes/erd_diagram.png}
%     \caption{Diagrama Entidad-Relación: relaciones principales entre User, Sede, Shift, Product, Payroll, Feedback.}
%     \label{fig:erd}
% \end{figure}

\section{Casos de Uso Principales}

Esta sección enumera los casos de uso más críticos del sistema, estructurados por rol:

\subsection*{CEO/Administrador}
\begin{enumerate}
    \item \textbf{UC1: Consultar Dashboard Ejecutivo:} CEO accede a resumen de KPIs (ventas global, ocupación plantilla por sede). Sistema filtra datos por período (hoy, semana, mes). Datos se actualizan en tiempo real vía WebSockets.
    
    \item \textbf{UC2: Dar de Alta Nueva Sede:} CEO rellena formulario (nombre, ubicación, manager asignado). Sistema valida que el manager no esté asignado a otra sede. Crea registro en DB y envía email de bienvenida al manager.
    
    \item \textbf{UC3: Revisar Feedback de Empleados:} CEO accede a tablero de feedback filtrable por sede, categoría, estado. Puede pasar de OPEN a IN\_REVIEW y luego RESOLVED.
\end{enumerate}

\subsection*{Gerente de Sede}
\begin{enumerate}
    \item \textbf{UC4: Asignar Turno a Empleado:} Gerente accede a calendario semanal. Selecciona empleado y fecha. Sistema valida que el empleado no tenga turno solapado. Asigna turno con estado PENDING (requiere confirmación de empleado vía app).
    
    \item \textbf{UC5: Gestionar Solicitud de Permuta:} Dos empleados intercambian turnos. Uno solicita (genera ShiftRequest). Gerente aprueba o rechaza. Si aprueba, los turnos se actualizan. Si rechaza, sigue en estado PENDING.
    
    \item \textbf{UC6: Crear Producto en Catálogo:} Gerente rellena formulario (nombre, precio, stock inicial). Sistema valida que nombre no sea duplicado en la sede. Genera product\_id autoincremental.
    
    \item \textbf{UC7: Ajustar Stock:} Gerente selecciona producto, incremente/decrementa cantidad. Sistema genera log de movimiento (auditoría). Si stock $<$ stock\_minimo, genera alerta visual.
    
    \item \textbf{UC8: Subir Nóminas:} Gerente selecciona período de mes y archivo PDF. Sistema valida que sea PDF. Vincula a employees por nombre/email. Notifica a empleado que su nómina está disponible.
\end{enumerate}

\subsection*{Empleado}
\begin{enumerate}
    \item \textbf{UC9: Consultar Próximo Turno:} Empleado abre dashboard. Sistema muestra próximo turno (fecha, hora, sede). Si turno en <24h, destaca en rojo.
    
    \item \textbf{UC10: Solicitar Cambio de Turno:} Empleado selecciona turno actual y turno deseado. Sistema valida disponibilidad del turno deseado. Crea ShiftRequest con status=PENDING. Notifica al gerente.
    
    \item \textbf{UC11: Ver Historial de Nóminas:} Empleado accede a sección ``Mis Nóminas''. Sistema lista todos los períodos con PDF descargable. Filtrable por año/mes.
    
    \item \textbf{UC12: Enviar Feedback:} Empleado rellena formulario con comentario, categoría (BUG/SUGGESTION/COMPLAINT). Sistema asigna status=OPEN y timestamp. Feedback viaja hacia CEO (visible en UC3).
\end{enumerate}

% TODO: DIAGRAMA DE CASOS DE USO UML
% \begin{figure}[h]
%     \centering
%     \includegraphics[width=0.85\textwidth]{imagenes/use_case_diagram.png}
%     \caption{Diagrama de casos de uso UML: actores (CEO, Manager, Employee) y relaciones con casos de uso principal del sistema.}
%     \label{fig:use_cases}
% \end{figure}

\section{Definición de Perfiles y Privilegios}

El sistema implementa un modelo de seguridad basado en roles (RBAC) jerárquico que garantiza el principio de mínimo privilegio: cada usuario accede únicamente a la información y funcionalidades estrictamente necesarias para el desempeño de su labor.

A continuación, se detalla el alcance funcional y los subsistemas accesibles para cada uno de los tres roles definidos en la organización. Cabe destacar que \textbf{toda la interfaz ha sido diseñada para ser ``Responsive''}, permitiendo su uso fluido tanto en ordenadores de escritorio como en tablets y dispositivos móviles, adaptando la disposición de los elementos al tamaño de la pantalla disponible.

\subsection{Subsistema de Dirección (Rol: CEO/Superadministrador)}

Este perfil dispone de una visión panorámica y estratégica de la organización. Su funcionalidad está orientada a la supervisión y al análisis comparativo, sin necesidad de intervenir en la operativa diaria de cada local.

\subsubsection{Dashboard Ejecutivo}

Panel de control principal con métricas en tiempo real y accesos rápidos. Actualización automática cada 30 segundos via WebSockets (sin necesidad de refrescar).

\begin{itemize}
    \item \textbf{Sección de Rendimiento (Arriba izquierda):}
    \begin{itemize}
        \item Gráfica de líneas: Ingresos acumulados (global) vs. período anterior. Descargable como CSV/PNG.
        \item Tabla comparativa de sedes: Ingresos, ocupación plantilla (\%), stock total. Filtrable y ordenable.
        \item Top 5 productos más vendidos (gráfica de barras) y productos con bajo stock (alerta visual).
    \end{itemize}
    
    \item \textbf{Sección de RRHH (Arriba derecha):}
    \begin{itemize}
        \item ``Headcount'': Total empleados activos vs. bajas (sick leave, vacation). Indicador de % de ocupación.
        \item Tablero de ausencias: Qué empleados están de baja hoy (click para detalles).
        \item Turnover rate: \% de rotación en últimos 12 meses.
    \end{itemize}
    
    \item \textbf{Atajos de Gestión (Abajo):}
    \begin{itemize}
        \item Botón para crear nueva sede (modal form).
        \item Listado de gerentes con contacto rápido (teléfono, email).
    \end{itemize}
\end{itemize}

\textbf{Validaciones y Reglas:} Datos mostrados siempre corresponden a 24 horas atrás (lag de 1 día) para mantener consistencia. CEO solo ve datos de sedes donde tiene autorización.

% TODO: MOCKUP DASHBOARD CEO
% \begin{figure}[h]
%     \centering
%     \includegraphics[width=0.95\textwidth]{imagenes/mockup_ceo_dashboard.png}
%     \caption{Mockup del Dashboard Ejecutivo: KPIs de rendimiento, RRHH y atajos de gestión. Diseño responsivo.}
%     \label{fig:mockup_ceo}
% \end{figure}

\subsubsection{Gestión de Infraestructura (Sedes)}

\textbf{Crear Nueva Sede:} Formulario con campos:
\begin{itemize}
    \item Nombre (requerido, máx 100 caracteres)
    \item Ubicación/dirección (requerida, máx 200 caracteres)
    \item Manager (dropdown de usuarios con rol MANAGER no asignados)
    \item Teléfono de contacto (opcional, formato internacional)
\end{itemize}

\textbf{Validaciones:}
\begin{itemize}
    \item El manager seleccionado no debe estar ya asignado a otra sede.
    \item Nombre de sede debe ser único dentro de la organización.
    \item Teléfono se valida con librería \texttt{libphonenumber-js}.
\end{itemize}

Al crear, sistema genera email de notificación al manager con credenciales de acceso y link a primer login.

\textbf{Editar/Ver Sedes:} Tabla con todas las sedes (2026 del año actual). Filtrable por nombre, manager. Expandible para ver detalles: empleados activos, productos, ventas del mes.

\subsubsection{Auditoría de Feedback}

Tablero consolidado de comentarios constructivos de empleados hacia CEO.

\begin{itemize}
    \item \textbf{Listado de Feedback:} Tabla con columnas: Empleado, Categoría (BUG/SUGGESTION/COMPLAINT), Estado (OPEN/IN\_REVIEW/RESOLVED), Fecha creación.
    \item \textbf{Filtros:} Por sede, categoría, estado, rango de fechas.
    \item \textbf{Detalles:} Click en fila abre modal con texto completo, posibilidad de cambiar estado.
\end{itemize}

\textbf{Estados y Transiciones:} OPEN → IN\_REVIEW (CEO marca como leído) → RESOLVED (CEO cierra después de accionar). Empleado notificado cuando su feedback pasa a RESOLVED.

\subsection{Subsistema de Gestión Local (Rol: Gerente de Sede)}

Es el perfil con mayor carga de interacción en el sistema. Gestiona los recursos humanos y materiales de \textbf{su ubicación asignada exclusivamente}. Todas las operaciones están filtradas por \texttt{sede\_id = gerente.sede\_id}.

\subsubsection{Dashboard del Gerente}

Panel operativo optimizado para decisiones rápidas.

\begin{itemize}
    \item \textbf{Resumen del Día Actual:}
    \begin{itemize}
        \item Empleados presentes hoy (foto, nombre, hora entrada esperada).
        \item Alertas urgentes: bajas imprevistas, bajo stock crítico, solicitudes pendientes de aprobación.
        \item Botones de acceso rápido: ``Asignar Turno'', ``Aprobar Permuta'', ``Crear Producto''.
    \end{itemize}
    
    \item \textbf{Mini-gráficas:}
    \begin{itemize}
        \item Ocupación plantilla esta semana (\% promedio).
        \item Stock total de productos (alerta si <20\% del mínimo).
        \item Ingresos del mes (comparativa con mes anterior).
    \end{itemize}
\end{itemize}

% TODO: MOCKUP DASHBOARD GERENTE
% \begin{figure}[h]
%     \centering
%     \includegraphics[width=0.95\textwidth]{imagenes/mockup_manager_dashboard.png}
%     \caption{Mockup Dashboard Gerente: resumen operativo del día, alertas y accesos rápidos a funciones más usadas.}
%     \label{fig:mockup_manager}
% \end{figure}

\subsubsection{Gestión de Plantilla (CRUD Empleados)}

\textbf{Alta de Empleado:} Formulario con campos:
\begin{itemize}
    \item Nombre completo, email, teléfono (requeridos)
    \item Puesto (dropdown: Vendedor, Supervisor, Operario)
    \item Horas contratadas/semana (requerido, rango 10-40)
    \item Fecha de inicio (requerida, mín. hoy)
\end{itemize}

\textbf{Validaciones:}
\begin{itemize}
    \item Email debe ser único en la organización.
    \item Horas contratadas entre 10 y 40 horas/semana.
    \item Sistema genera \texttt{user\_id} autoincremental y contraseña temporal (enviada por email).
\end{itemize}

\textbf{Edición:} Permite cambiar puesto, horas contratadas. Cambios auditorados en log (quién, cuándo, qué cambió).

\textbf{Baja/Inactivación:} Marca empleado como \texttt{status=INACTIVE}. Su sueldo deja de acumularse desde esa fecha. Turnos futuros se liberan.

\subsubsection{Planificador y Gestión de Turnos}

\textbf{Flujo de Asignación de Turno (UC4):}

\begin{enumerate}
    \item Gerente accede a vista de calendario semanal/mensual (drag-and-drop).
    \item Selecciona un empleado y un día.
    \item Especifica hora entrada y hora salida (ambas requeridas, no pueden ser iguales).
    \item Sistema valida:
    \begin{itemize}
        \item Empleado está activo (status=ACTIVE).
        \item No hay turno solapado (overlap detection): \texttt{(new.start < existing.end) AND (new.end > existing.start)}.
        \item Turno no va más allá de horas contratadas/semana.
    \end{itemize}
    \item Si validaciones pasan: turno se crea con estado \texttt{status=PENDING}.
    \item Empleado recibe notificación push/email: ``Nuevo turno asignado [fecha, horas]. Confírma tu asistencia''.
\end{enumerate}

\textbf{Estados de Turno:}
\begin{itemize}
    \item \texttt{PENDING}: Recién asignado, espera confirmación de empleado.
    \item \texttt{CONFIRMED}: Empleado confirmó asistencia.
    \item \texttt{WORKED}: Turno ya pasó (fechado en el pasado).
    \item \texttt{CANCELLED}: Anulado (por gerente o empleado con causa).
    \item \texttt{NO\_SHOW}: Empleado no se presentó sin justificación.
\end{itemize}

\textbf{Vista Diaria:} Tabla con turnos de hoy, mostrando: Empleado, Hora entrada/salida, Estado, Acciones (editar/cancelar si aún no confirmado).

% TODO: FLOWCHART DE ASIGNACIÓN DE TURNO
% \begin{figure}[h]
%     \centering
%     \includegraphics[width=0.7\textwidth]{imagenes/flowchart_shift_assignment.png}
%     \caption{Flujo de validación y asignación de turno: detección de solapamientos, validación de horas, cambio de estado.}
%     \label{fig:flowchart_shift}
% \end{figure}

\textbf{Solicitudes de Permuta (UC5):}

Dos empleados quieren intercambiar turnos.

\begin{enumerate}
    \item Empleado A solicita permuta desde su app: selecciona su turno (T1) y propone intercambiar con turno de B (T2).
    \item Sistema valida: B no tiene conflicto si toma T1. A no tiene conflicto si toma T2.
    \item Crea \texttt{ShiftRequest} con \texttt{status=PENDING}.
    \item Gerente ve en ``Centro de Notificaciones'' solicitud de permuta. Puede:
    \begin{itemize}
        \item \textbf{Aprobar:} Turnos se intercambian. Ambos empleados notificados.
        \item \textbf{Rechazar:} ShiftRequest.status = REJECTED. Empleados notificados.
    \end{itemize}
\end{enumerate}

\textbf{Centro de Notificaciones:} Panel desplegable con:
\begin{itemize}
    \item Solicitudes de permutas pendientes (con avatares de empleados).
    \item Solicitudes de baja/permiso (si implementado).
    \item Alertas de operación (ej: empleado confirmó ausencia, nuevo empleado dado de alta).
\end{itemize}

\subsubsection{Gestión de Catálogo e Inventario}

\textbf{CRUD de Productos (UC6):}

Crear: Formulario con Nombre, Descripción (opcional), Precio venta, Cantidad inicial. Sistema valida nombre único por sede. Genera \texttt{product\_id}.

Editar: Permite cambiar precio y descripción. Cambios se auditorean.

Eliminar: Marca \texttt{status=DELETED} (soft delete). No aparece en catálogo pero se preserva en histórico de ventas.

% TODO: MOCKUP CATÁLOGO PRODUCTOS
% \begin{figure}[h]
%     \centering
%     \includegraphics[width=0.85\textwidth]{imagenes/mockup_catalog.png}
%     \caption{Mockup sección de Catálogo: tabla de productos con precios, stock, acciones (edit/delete).}
%     \label{fig:mockup_catalog}
% \end{figure}

\textbf{Control de Stock (UC7):}

\begin{itemize}
    \item \textbf{Ajuste Manual:} Gerente selecciona producto, ingresa cantidad a ajustar (+ o -). Sistema genera log: \texttt{Ajuste: [producto], [cantidad], [motivo], [usuario], [timestamp]}.
    
    \item \textbf{Alertas de Stock:} Si \texttt{stock\_actual < stock\_minimo}, icono de alerta en rojo en dashboard. Gerente puede reordenar (funcionalidad futura de integración POS).
    
    \item \textbf{Inventario:} Opción ``Realizar Inventario'': Gerente cuenta productos físicos y ajusta cantidades en sistema (puede tomar foto como comprobante).
\end{itemize}

\textbf{Validaciones:}
\begin{itemize}
    \item Stock nunca puede ser negativo.
    \item Ajustes requieren motivo (selección de lista dropdown: Venta, Rotura, Aportación, Corrección, etc.).
\end{itemize}

\subsubsection{Gestión Documental (Nóminas)}

\textbf{Subida de Nóminas (UC8):}

\begin{enumerate}
    \item Gerente accede a sección ``Nóminas''.
    \item Selecciona período (mes/año de dropdown).
    \item Sube archivo PDF. Sistema valida:
    \begin{itemize}
        \item Es PDF (no otra extensión).
        \item Tamaño <10MB.
    \end{itemize}
    \item Gerente selecciona de dropdown qué empleado es destinatario (o sube múltiples nóminas a la vez con asignación automática por nombre).
    \item Sistema crea registro \texttt{Payroll} e inmediatamente notifica al empleado: ``Tu nómina de [mes] está disponible''.
\end{enumerate}

\textbf{Auditoría:} Cada PDF se almacena encriptado (AES-256) en almacenamiento seguro. Acceso se loguea.

\subsection{Subsistema Operativo (Rol: Empleado)}

Diseñado con una interfaz simplificada y \textit{Mobile-First}, este perfil actúa como un consumidor de información y emisor de feedback. Todas las pantallas se cargan en <2 segundos en conexión 3G.

\subsubsection{Dashboard del Empleado}

Pantalla inicial (home screen) personalizada para cada empleado.

\begin{itemize}
    \item \textbf{Tarjeta de Próximo Turno (Destacada):}
    \begin{itemize}
        \item Fecha, hora entrada/salida de turno más próximo.
        \item Si turno es hoy: botón ``Confirmar Asistencia'' (debe presionarlo antes del turno).
        \item Si turno es en <24h: se destaca en amarillo. Si es en <1h: en rojo.
        \item Botón ``+ Info'': detalles de ubicación sede, contacto gerente.
    \end{itemize}
    
    \item \textbf{Notificaciones Relevantes:}
    \begin{itemize}
        \item Nuevo turno asignado (gerente lo acaba de crear).
        \item Permuta aprobada/rechazada.
        \item Nueva nómina disponible.
    \end{itemize}
    
    \item \textbf{Acceso Rápido:}
    \begin{itemize}
        \item Botón a ``Mis Turnos'' (calendario).
        \item Botón a ``Mis Nóminas''.
        \item Botón a ``Enviar Feedback''.
    \end{itemize}
\end{itemize}

% TODO: MOCKUP DASHBOARD EMPLEADO MÓVIL
% \begin{figure}[h]
%     \centering
%     \includegraphics[width=0.5\textwidth]{imagenes/mockup_employee_dashboard_mobile.png}
%     \caption{Mockup Dashboard Empleado (móvil): próximo turno, notificaciones, accesos rápidos. Optimizado para pantalla pequeña.}
%     \label{fig:mockup_employee}
% \end{figure}

\subsubsection{Calendario y Turnos}

\textbf{Vistas de Calendario (UC9):}

\begin{itemize}
    \item \textbf{Vista Semanal (default):} Grilla con 7 días. Cada celda muestra turno (si existe) con hora. Color verde = CONFIRMED, amarillo = PENDING, gris = WORKED (pasado).
    
    \item \textbf{Vista Mensual:} Calendario clásico. Cada día con pequeño indicador (punto) si tiene turno. Click en día expande para ver detalles.
    
    \item \textbf{Filtros:} Sede (si aplica), estado de turno.
\end{itemize}

\textbf{Solicitud de Cambio de Turno (UC10):}

Flujo de permuta desde perspectiva de empleado:

\begin{enumerate}
    \item Empleado selecciona su turno (T1) desde calendario. Aparece botón ``Solicitar Permuta''.
    \item Ventana emerge permitiendo seleccionar turno de otro compañero (T2). Sistema muestra solo turnos de compañeros disponibles ese día (búsqueda filtrada).
    \item Empleado confirma. Sistema valida que ambos turnos sean compatibles.
    \item \texttt{ShiftRequest} se crea con \texttt{status=PENDING}. Aparece en notificaciones de empleado como ``En Espera de Aprobación''.
    \item Mientras solicitud esté pending, no puede refundir new requests (evita spam).
    \item Cuando gerente aprueba/rechaza, empleado recibe notificación push inmediata.
\end{enumerate}

\subsubsection{Mis Nóminas}

\textbf{Visualización y Descarga (UC11):}

\begin{itemize}
    \item \textbf{Listado de Nóminas:} Tabla con columnas: Período (Ej: ``Enero 2026''), Fecha disponible, Acciones (Descargar).
    \item \textbf{Filtro por Año:} Dropdown para cambiar año y ver histórico completo.
    \item \textbf{Descargar:} Click en botón descarga PDF (encriptado en almacenamiento, se descifra al descargar).
    \item \textbf{Seguridad:} Empleado SOLO ve sus propias nóminas. Queryse filtra por \texttt{user\_id = logged\_user.id}.
\end{itemize}

\subsubsection{Sistema de Feedback}

\textbf{Envío de Feedback (UC12):}

\begin{itemize}
    \item \textbf{Formulario:} Campos:
    \begin{itemize}
        \item ``¿De qué es tu comentario?'' (radio button): BUG / SUGGESTION / COMPLAINT.
        \item ``Tu mensaje'' (textarea, máx 500 caracteres).
    \end{itemize}
    
    \item \textbf{Validaciones:}
    \begin{itemize}
        \item Mensaje no puede estar vacío.
        \item Máximo 3 feedbacks por día (throttling para evitar spam).
    \end{itemize}
    
    \item \textbf{Al Enviar:}
    \begin{itemize}
        \item Sistema crea registro \texttt{Feedback} con \texttt{status=OPEN}.
        \item Toast de confirmación: ``Gracias por tu feedback. Lo recibirá nuestro equipo''.
        \item CEO lo ve en su tablero de Auditoría (UC3).
    \end{itemize}
\end{itemize}

% TODO: MOCKUP FORMULARIO FEEDBACK
% \begin{figure}[h]
%     \centering
%     \includegraphics[width=0.55\textwidth]{imagenes/mockup_feedback_form.png}
%     \caption{Mockup Formulario de Feedback: simple, mobile-friendly, claro.}
%     \label{fig:mockup_feedback}
% \end{figure}

\section{Sistema de Notificaciones y Alertas}

Las notificaciones son críticas para mantener usuarios informados en tiempo real.

\subsection*{Canales de Notificación}

\begin{itemize}
    \item \textbf{Push Notifications (In-App):} Aparecen en esquina superior de pantalla. Auto-desaparecen en 5 segundos. Click lleva a pantalla relevante.
    
    \item \textbf{Email:} Eventos importantes (nuevo turno, permuta rechazada, nómina disponible). Enviados vía servicio SMTP (SendGrid o similar). \textbf{No spam}: máximo 1 email por día por tipo de evento.
    
    \item \textbf{En-App Badge:} Ícono de campana con contador (ej: ``3 notificaciones nuevas'').
\end{itemize}

\subsection*{Eventos que Generan Notificaciones}

\begin{table}[h]
    \centering
    \caption{Matriz de notificaciones: qué rol recibe qué evento por qué canal}
    \label{tab:notifications}
    \vspace{0.3cm}
    \begin{tabularx}{\textwidth}{l l X}
        \toprule
        \textbf{Evento} & \textbf{Rol} & \textbf{Canales} \\
        \midrule
        Nuevo turno asignado & Empleado & Push + Email \\
        Turno confirmado/rechazado & Gerente & Push \\
        Solicitud de permuta recibida & Gerente & Push \\
        Permuta aprobada/rechazada & Empleado & Push + Email \\
        Nómina disponible & Empleado & Push + Email \\
        Nuevo feedback recibido & CEO & Push \\
        Stock bajo crítico & Gerente & Push \\
        Nueva sede creada & Gerente (assigned) & Email \\
        \bottomrule
    \end{tabularx}
\end{table}

\section{Flujos de Navegación y Menú}

Cada rol dispone de un menú lateral (en desktop) o hamburguesa (en móvil) con secciones accesibles.

\subsection*{CEO}
\begin{itemize}
    \item Dashboard Ejecutivo
    \item Gestión de Sedes
    \item Auditoría de Feedback
    \item Configuración (cambiar contraseña, idioma, zona horaria)
    \item Logout
\end{itemize}

\subsection*{Gerente}
\begin{itemize}
    \item Dashboard Operativo
    \item Planificador de Turnos
    \item Gestión de Empleados
    \item Catálogo e Inventario
    \item Nóminas
    \item Centro de Notificaciones
    \item Mi Perfil
    \item Logout
\end{itemize}

\subsection*{Empleado}
\begin{itemize}
    \item Mis Turnos (Calendario)
    \item Mis Nóminas
    \item Enviar Feedback
    \item Mi Perfil (editar teléfono, email, contraseña)
    \item Logout
\end{itemize}

\section{Integraciones y Sincronización en Tiempo Real}

\subsection*{Sincronización de Datos Entre Capas}

Las operaciones críticas emplean \textbf{transacciones ACID} en PostgreSQL para garantizar consistencia:

\begin{itemize}
    \item \textbf{Al asignar turno:} En una única transacción se actualiza \texttt{Shift.status}, se decrementa horas disponibles del empleado, se incrementa una columna de auditoría. Si falla cualquier paso, ROLLBACK.
    
    \item \textbf{Al aprobar permuta:} Transacción que intercambia IDs de \texttt{shift.employee\_id} de ambos turnos de forma atómica. Imposible que quede estado inconsistente.
\end{itemize}

\subsection*{Actualizaciones en Tiempo Real para CEO}

El Dashboard Ejecutivo se actualiza automáticamente sin que CEO necesite refrescar:

\begin{itemize}
    \item Backend emite eventos vía WebSocket cuando cambian datos relevantes: nuevo turno creado, nómina subida, feedback enviado.
    \item Frontend mantiene conexión WebSocket abierta. Al recibir evento, actualiza gráficas y tablas (sin molestar si CEO está interactuando).
    \item Fallback: Si WebSocket se cae, frontend hace polling cada 30 segundos.
\end{itemize}

\section{Privacidad y Consideraciones de Seguridad}

\subsection*{Filtrado de Datos}

\textbf{Principio:} Cada usuario SOLO ve datos según su rol:

\begin{itemize}
    \item \textbf{Empleado:} Ve solo sus propios turnos, su nómina, su feedback.
    \item \textbf{Gerente:} Ve empleados, turnos, inventario de su sede solamente. No ve datos de otras sedes.
    \item \textbf{CEO:} Ve agregados (suma, promedio) pero no datos individuales (ej: no ve nombre de empleado en turno, solo código employeeID si necesario).
\end{itemize}

\textbf{Implementación:} Middleware Express valida cada query: \texttt{WHERE sede\_id = user.sede\_id OR user.role = ADMIN}. Queries SQL parameterizadas previenen inyecciones.

\subsection*{Protección de Datos Sensibles}

\begin{itemize}
    \item \textbf{Nóminas:} PDFs encriptados (AES-256) en almacenamiento. Descifrado solo en lado cliente, nunca transmitido en claro.
    \item \textbf{Contraseñas:} Hasheadas con bcrypt (10 rounds). No se almacena ni se recupera texto plano.
    \item \textbf{Datos de contacto (email/teléfono):} Accesibles solo a roles autorizados (gerente ve teléfono de sus empleados, CEO no).
    \item \textbf{HTTPS obligatorio:} Todas las comunicaciones con TLS 1.2+. HSTS header configurado.
\end{itemize}

\subsection*{Auditoría}

Operaciones críticas se registran en tabla \texttt{AuditLog}:

\begin{itemize}
    \item Quién (user\_id), Qué (operación), Cuándo (timestamp), Dónde (tabla afectada).
    \item Ejemplos: ``Manager user\_42 creó turno para employee\_15, 2026-02-09 10:30'', ``emp\_18 descaró nómina enero, 2026-02-09 14:45''.
    \item Acceso a auditlog restringido a CEO (solo para su sede).
\end{itemize}

\section{Testing de Funcionalidades Críticas}

La especificación funcional anterior debe validarse mediante tests:

\subsection*{Unit Tests Recomendados}
\begin{itemize}
    \item Validación de overlap de turnos (función pura).
    \item Cálculo de horas totales/semana.
    \item Reglas de permisos RBAC.
\end{itemize}

\subsection*{Integration Tests Recomendados}
\begin{itemize}
    \item Flujo completo asignación turno: crear → validar → guardar → notificar.
    \item Flujo aprobación permuta: ambos turnos intercambian correctamente.
    \item Subida nómina: archivo guardado, empleado notificado, accesible en app.
\end{itemize}

\subsection*{E2E Tests Recomendados}
\begin{itemize}
    \item CEO login → ver dashboard → crear sede → verificar aparece.
    \item Gerente login → asignar turno → empleado ve notificación.
    \item Empleado solicita permuta → gerente aprueba → ambos ven cambios.
\end{itemize}

\section{Consideraciones para Futuras Extensiones}

El diseño modular permite agregar funcionalidades sin afectar lo existente:

\begin{itemize}
    \item \textbf{Sistema de Calificaciones:} CEO y Gerentes puntúan empleados (1-5 estrellas). Permite reestructuración de equipos.
    \item \textbf{Integración POS:} Conectar con caja registradora para registrar ventas automáticamente (hoy es manual).
    \item \textbf{Horas Extras y Suplementos:} Cálculos de compensación si turno excede horas contratadas.
    \item \textbf{Roles Adicionales:} Supervisor, Coordinador (entre Manager y CEO).
    \item \textbf{Aplicación Móvil Nativa:} Mismo backend API, cliente iOS/Android con React Native.
    \item \textbf{Integración HR Compliance:} Conectar con software de recursos humanos (Factorial, BizMérida) para sincronizar empleados.
\end{itemize}

% ==========================================================
% CAPÍTULO 5: PLAN DE DESARROLLO ÁGIL Y DIVISIÓN DE TAREAS
% ==========================================================
\chapter{Plan de Desarrollo Ágil}
\label{ch:desarrollo}

\section{Metodología de Desarrollo}

El proyecto se desarrolla bajo \textbf{metodología Scrum adaptada} con ciclos iterativos de 2 semanas.

\subsection*{Parámetros del Proyecto}

\begin{itemize}
    \item \textbf{Duración Total:} 12 semanas (febrero 9 - mayo 4, 2026)
    \item \textbf{Dedicación:} 15 horas/semana (promedio)
    \item \textbf{Total de Horas Disponibles:} 180 horas
    \item \textbf{Número de Sprints:} 6 sprints de 2 semanas
    \item \textbf{Horas/Sprint:} ~30 horas (2 sprints paralelos = 60h/mes)
\end{itemize}

\section{Product Backlog Priorizado}

El Product Backlog contiene todas las funcionalidades organizadas por prioridad, estimadas en \textbf{Story Points}.


\subsection*{Grupo 1: Infraestructura y Seguridad (CRÍTICO - Sprint 1-2)}

\begin{table}[h]
    \centering
    \caption{Backlog: Infraestructura}
    \vspace{0.3cm}
    \begin{tabularx}{\textwidth}{l l c l}
        \toprule
        \textbf{ID} & \textbf{Descripción} & \textbf{SP} & \textbf{Prioridad} \\
        \midrule
        PBI-1 & Setup: git, ESLint, Prettier, carpetas & 3 & ALTA \\
        PBI-2 & PostgreSQL + schema inicial (User, Sede, Shift) & 3 & ALTA \\
        PBI-3 & Express: servidor, middleware, CORS & 2 & ALTA \\
        PBI-4 & React + Vite + Tailwind + componentes base & 3 & ALTA \\
        PBI-5 & JWT: access/refresh tokens + logout & 5 & ALTA \\
        PBI-6 & Endpoint /auth/login + validaciones & 3 & ALTA \\
        PBI-7 & Pantalla Login (frontend) & 3 & ALTA \\
        PBI-8 & RBAC middleware (CEO/Manager/Employee) & 5 & ALTA \\
        PBI-9 & Endpoint /auth/register (solo admin) & 2 & ALTA \\
        \bottomrule
    \end{tabularx}
\end{table}

\subsection*{Grupo 2: Gestión de Usuarios (Sprint 2-3)}

\begin{table}[h]
    \centering
    \caption{Backlog: Usuarios y Sedes}
    \vspace{0.3cm}
    \begin{tabularx}{\textwidth}{l l c l}
        \toprule
        \textbf{ID} & \textbf{Descripción} & \textbf{SP} & \textbf{Prioridad} \\
        \midrule
        PBI-10 & Endpoints CRUD usuarios (GET/POST/PUT/DELETE) & 5 & ALTA \\
        PBI-11 & Endpoints CRUD sedes & 3 & ALTA \\
        PBI-12 & Validaciones Zod (usuario, sede) & 2 & ALTA \\
        PBI-13 & Pantalla Admin: Crear usuario + asignar rol/sede & 5 & ALTA \\
        PBI-14 & Pantalla Admin: Listar usuarios, filtrar, editar & 5 & ALTA \\
        \bottomrule
    \end{tabularx}
\end{table}

\subsection*{Grupo 3: Dashboard CEO (Sprint 3-4)}

\begin{table}[h]
    \centering
    \caption{Backlog: Dashboard Ejecutivo}
    \vspace{0.3cm}
    \begin{tabularx}{\textwidth}{l l c l}
        \toprule
        \textbf{ID} & \textbf{Descripción} & \textbf{SP} & \textbf{Prioridad} \\
        \midrule
        PBI-15 & Endpoint GET /dashboard/metrics (ventas, ocupación, stock) & 8 & ALTA \\
        PBI-16 & Pantalla Dashboard: layout KPIs básico & 5 & ALTA \\
        PBI-17 & Gráficas Recharts (líneas, barras, tablas) & 5 & MEDIA \\
        PBI-18 & TanStack Query: caché de datos en frontend & 3 & MEDIA \\
        PBI-19 & Filtros por fecha y sede & 3 & MEDIA \\
        \bottomrule
    \end{tabularx}
\end{table}

\subsection*{Grupo 4: Gestión de Turnos (Sprint 3-5) - CORE}

\begin{table}[h]
    \centering
    \caption{Backlog: Turnos}
    \vspace{0.3cm}
    \begin{tabularx}{\textwidth}{l l c l}
        \toprule
        \textbf{ID} & \textbf{Descripción} & \textbf{SP} & \textbf{Prioridad} \\
        \midrule
        PBI-20 & Endpoints CRUD turnos (crear, listar, cancelar) & 8 & ALTA \\
        PBI-21 & Validación overlaps + unit tests & 5 & ALTA \\
        PBI-22 & Endpoint confirmar turno (empleado) & 3 & ALTA \\
        PBI-23 & Calendario Gerente: semanal + crear turno & 8 & ALTA \\
        PBI-24 & Calendario Empleado: semanal/mensual & 5 & ALTA \\
        \bottomrule
    \end{tabularx}
\end{table}

\subsection*{Grupo 5: Permutas de Turnos (Sprint 5)}

\begin{table}[h]
    \centering
    \caption{Backlog: Solicitudes de Permuta}
    \vspace{0.3cm}
    \begin{tabularx}{\textwidth}{l l c l}
        \toprule
        \textbf{ID} & \textbf{Descripción} & \textbf{SP} & \textbf{Prioridad} \\
        \midrule
        PBI-25 & Endpoint crear/listar ShiftRequest & 5 & MEDIA \\
        PBI-26 & Endpoint aprobar/rechazar permuta & 5 & MEDIA \\
        PBI-27 & Pantalla Gerente: Centro de Notificaciones & 5 & MEDIA \\
        PBI-28 & Pantalla Empleado: Solicitar permuta & 5 & MEDIA \\
        \bottomrule
    \end{tabularx}
\end{table}

\subsection*{Grupo 6: Inventario (Sprint 5-6)}

\begin{table}[h]
    \centering
    \caption{Backlog: Gestión de Productos}
    \vspace{0.3cm}
    \begin{tabularx}{\textwidth}{l l c l}
        \toprule
        \textbf{ID} & \textbf{Descripción} & \textbf{SP} & \textbf{Prioridad} \\
        \midrule
        PBI-29 & Endpoints CRUD productos (crear, editar, listar) & 5 & MEDIA \\
        PBI-30 & Endpoint ajustar stock + log cambios & 5 & MEDIA \\
        PBI-31 & Alertas bajo stock (lógica backend) & 2 & MEDIA \\
        PBI-32 & Pantalla Gerente: Catálogo productos & 5 & MEDIA \\
        PBI-33 & Pantalla Gerente: Ajustar stock & 3 & MEDIA \\
        \bottomrule
    \end{tabularx}
\end{table}

\subsection*{Grupo 7: Nóminas (Sprint 6 - OPCIONAL)}

\begin{table}[h]
    \centering
    \caption{Backlog: Gestión de Nóminas}
    \vspace{0.3cm}
    \begin{tabularx}{\textwidth}{l l c l}
        \toprule
        \textbf{ID} & \textbf{Descripción} & \textbf{SP} & \textbf{Prioridad} \\
        \midrule
        PBI-34 & Endpoint subir PDF nómina (validación básica) & 5 & BAJA \\
        PBI-35 & Endpoint listar nóminas empleado & 2 & BAJA \\
        PBI-36 & Pantalla Gerente: Subir nóminas & 3 & BAJA \\
        PBI-37 & Pantalla Empleado: Mis nóminas + descarga & 3 & BAJA \\
        \bottomrule
    \end{tabularx}
\end{table}


\subsection*{Grupo 8: Feedback y Testing (Sprint 6 - OPCIONAL)}

\begin{table}[h]
    \centering
    \caption{Backlog: Feedback y Testing}
    \vspace{0.3cm}
    \begin{tabularx}{\textwidth}{l l c l}
        \toprule
        \textbf{ID} & \textbf{Descripción} & \textbf{SP} & \textbf{Prioridad} \\
        \midrule
        PBI-38 & Endpoint feedback (crear, listar, cambiar estado) & 5 & BAJA \\
        PBI-39 & Pantalla Empleado: Formulario feedback & 2 & BAJA \\
        PBI-40 & Pantalla CEO: Tablero feedback & 3 & BAJA \\
        PBI-41 & Unit tests: 50+ tests, 80\% cobertura & 13 & BAJA \\
        PBI-42 & Integration tests: 5+ flujos críticos & 8 & BAJA \\
        \bottomrule
    \end{tabularx}
\end{table}

\section{Planificación de Sprints (6 x 2 semanas)}

\subsection*{Sprint 1 (Feb 9-22): Infraestructura y Autenticación}
\textbf{PBIs:} 1-9 (28 SP) \textbf{COMPLETADO}

El Sprint 1 constituye la fase fundamental del proyecto, estableciendo los cimientos arquitectónicos y sistemas de autenticación requeridos para el funcionamiento seguro de la aplicación. Este sprint se centró en la implementación de una infraestructura robusta que cumpliera con estándares industry e implementara buenas prácticas de seguridad.

\subsubsection*{PBI-1: Setup (git, ESLint, Prettier)}

La inicialización del proyecto mediante control de versiones y definición de estándares de código es esencial para garantizar la mantenibilidad y colaboración eficiente \cite{typescript_docs}. Se realizó la configuración de:

\begin{itemize}
    \item \textbf{Gestor de Control de Versiones:} Repositorio Git en GitHub (TFG-Multihub) para tracking de cambios y colaboración
    \item \textbf{Linting:} ESLint con configuración \textit{airbnb-typescript} para mantener consistencia de código y prevenir errores comunes
    \item \textbf{Formateo:} Prettier configurado para garantizar formato automático coherente en todo el codebase
    \item \textbf{Estructura Monorepo:} Directorios separados para \texttt{/backend} y \texttt{/frontend} permitiendo gestión independiente de dependencias
\end{itemize}

\subsubsection*{PBI-2: PostgreSQL + Schema de Base de Datos}

El diseño relacional del sistema requería una base de datos que proporcionara ACID compliance y transaccionalidad robusta. Se seleccionó PostgreSQL 16 como motor de almacenamiento, complementado con Prisma ORM \cite{prisma_docs} para abstracción de datos tipada en TypeScript.

El schema relacional implementado define los siguientes modelos de datos:

\begin{itemize}
    \item \textbf{User}: Entidad representante de usuarios del sistema con campos de autenticación, identificación y asignación de roles (ADMIN, MANAGER, EMPLOYEE)
    \item \textbf{Sede}: Entidades de oficinas con información de localización, contacto y manager responsable
    \item \textbf{Shift}: Entidades de turnos laborales con validación de no-solapamiento mediante restricciones de base de datos
    \item \textbf{ShiftRequest}: Transacciones de solicitud de permuta entre empleados con workflow de aprobación
\end{itemize}

Ejemplo del schema en Prisma:
\begin{lstlisting}[language=typescript]
model User {
  id        String   @id @default(cuid())
  email     String   @unique
  password  String
  name      String
  role      UserRole
  sedeId    String?
  createdAt DateTime @default(now())
  // ... relaciones y timestamps ...
}

enum UserRole {
  ADMIN
  MANAGER
  EMPLOYEE
}

model Shift {
  id       String   @id @default(cuid())
  type     ShiftType
  date     DateTime
  startTime String
  endTime  String
  // Restricción única para prevenir overlaps
  @@unique([date, type, employeeId, sedeId])
}
\end{lstlisting}

Justificación: El uso de Prisma ORM proporciona type-safety en tiempo de compilación, migraciones versionadas, y validaciones a nivel de schema. PostgreSQL fue seleccionado por su fiabilidad comprobada en entornos empresariales \cite{postgresql_docs} y soporte nativo de tipos de datos avanzados.

\subsubsection*{PBI-3: Express.js + Middleware}

El backend fue implementado usando Express.js \cite{express_docs}, framework minimalista pero robusto de Node.js. La configuración incluyó:

\begin{itemize}
    \item \textbf{Middleware CORS:} Configuración enabling credenciales para intercambio seguro de tokens JWT y cookies \cite{cors_mdn}
    \item \textbf{Body Parser:} Middleware para parseado automático de payloads JSON
    \item \textbf{Error Handling:} Middleware global para captura y normalización de errores
    \item \textbf{Health Check:} Endpoint de diagnosticidad en \texttt{GET /health} para monitoreo de infraestructura
\end{itemize}

La configuración de CORS fue específicamente diseñada para permitir credenciales, prerequisito esencial para validación de refresh tokens en httpOnly cookies:

\begin{lstlisting}[language=typescript]
app.use(cors({
  origin: process.env.FRONTEND_URL || 'http://localhost:3000',
  credentials: true,
  methods: ['GET', 'POST', 'PUT', 'DELETE', 'PATCH'],
  allowedHeaders: ['Content-Type', 'Authorization'],
}));
\end{lstlisting}

Justificación: La habilitación de credenciales es crítica en arquitecturas que utilizan httpOnly cookies para almacenar refresh tokens, previniendo exposición a ataques XSS al no permitir acceso vía JavaScript \cite{cors_mdn}.

\subsubsection*{PBI-4: React + Vite + Tailwind}

El frontend fue construido con stack moderno basado en React 18 \cite{react_docs}, optimizando para desarrollo rápido y compilación eficiente:

\begin{itemize}
    \item \textbf{Bundler:} Vite para compilación ultra-rápida con HMR (Hot Module Replacement)
    \item \textbf{Estilos:} Tailwind CSS para componentes styled-components basados en utilidades
    \item \textbf{Navegación:} React Router para SPA con rutas públicas y protegidas
    \item \textbf{Gestión de Datos:} TanStack Query para caching y sincronización con servidor
    \item \textbf{Tipado:} TypeScript strict mode para máxima seguridad de tipos \cite{typescript_docs}
\end{itemize}

Justificación: La selección de Vite sobre Webpack proporciona tiempos de dev server 10-100x más rápidos, mejorando significativamente la experiencia de desarrollo. TypeScript en strict mode captura errores en compile-time que de otro modo se manifestarían en runtime.

\subsubsection*{PBI-5: JSON Web Tokens (JWT) - Tokens de Acceso y Refresca}

El sistema de autenticación fue implementado siguiendo el estándar RFC 7519 \cite{rfc7519} para JSON Web Tokens. La arquitectura utiliza un modelo de dos tokens para equilibrar seguridad con usabilidad:

\begin{itemize}
    \item \textbf{Access Token}: Duración 15 minutos. Utilizado para autenticación de requests API. Su corta duración limita exposición en caso de compromise.
    \item \textbf{Refresh Token}: Duración 7 días. Almacenado en httpOnly cookie para prevención de ataques XSS. Requerido para obtención de nuevos access tokens.
\end{itemize}

Las contraseñas de usuario son hasheadas usando bcryptjs \cite{bcryptjs_docs} con cost factor de 10 rounds, proporcionando resistencia computacional contra ataques de fuerza bruta.

Implementación de generación de tokens:

\begin{lstlisting}[language=typescript]
export function generateTokens(
  userId: string, 
  email: string, 
  role: string
) {
  const accessToken = jwt.sign(
    { userId, email, role },
    process.env.JWT_SECRET!,
    { expiresIn: '15m' }
  );

  const refreshToken = jwt.sign(
    { userId, email, role },
    process.env.JWT_REFRESH_SECRET!,
    { expiresIn: '7d' }
  );

  return { accessToken, refreshToken };
}
\end{lstlisting}

Middleware de validación y extracción de claims:

\begin{lstlisting}[language=typescript]
export const authMiddleware = (
  req: Request, 
  res: Response, 
  next: NextFunction
) => {
  const token = req.headers.authorization?.split(' ')[1];

  if (!token) {
    return res.status(401).json({ error: 'No token provided' });
  }

  try {
    const payload = jwt.verify(token, process.env.JWT_SECRET!);
    req.user = payload;
    next();
  } catch (error) {
    res.status(401).json({ error: 'Invalid token' });
  }
};
\end{lstlisting}

Justificación: El modelo de dos tokens es recomendación estándar en OAuth 2.0 \cite{oauth2_rfc} para aplicaciones modernas, balanceando seguridad (access token corto) con experiencia de usuario (refresh token permite sesiones largas sin re-login). El almacenamiento de refresh tokens en httpOnly cookies previene acceso vía JavaScript, mitigando XSS attacks.

\subsubsection*{PBI-6: Endpoint /auth/login}

El endpoint de autenticación implementa flujo de validación de credenciales con las siguientes etapas:

\begin{lstlisting}[language=typescript]
async login(credentials: LoginRequest): Promise<LoginResponse> {
  // Validación de esquema usando Zod
  const validated = LoginSchema.parse(credentials);
  
  // Búsqueda de usuario por email
  const user = await prisma.user.findUnique({
    where: { email: validated.email },
  });

  if (!user) {
    throw new Error('Usuario o contraseña incorrectos');
  }

  // Comparación hasheada de contraseña (previene timing attacks)
  const passwordMatch = await bcryptjs.compare(
    validated.password, 
    user.password
  );

  if (!passwordMatch) {
    throw new Error('Usuario o contraseña incorrectos');
  }

  // Generación de JWT tokens
  const tokens = generateTokens(user.id, user.email, user.role);
  
  return {
    user: this.excludePassword(user),
    tokens,
  };
}
\end{lstlisting}

Características de seguridad implementadas:
\begin{itemize}
    \item \textbf{Constant-time Comparison}: Uso de \texttt{bcryptjs.compare()} previene timing attacks
    \item \textbf{Mensajes de Error Genéricos}: No se especifica si el email existe, previniendo enumeración de usuarios
    \item \textbf{Exclusión de Contraseña}: La contraseña nunca es retornada al cliente
\end{itemize}

\subsubsection*{PBI-7: Componente de Login (Frontend)}

El componente de autenticación frontend implementa un formulario con validaciones en tiempo real y gestión de estado:

\begin{lstlisting}[language=typescript]
export const AuthProvider: React.FC<{ children: React.ReactNode }> = 
({ children }) => {
  const [user, setUser] = useState<User | null>(getStoredUser());
  const [isLoading, setIsLoading] = useState(false);

  const login = async (credentials: LoginRequest): Promise<void> => {
    setIsLoading(true);
    try {
      const response = await authAPI.login(credentials);
      setAccessToken(response.data.tokens.accessToken);
      setStoredUser(response.data.user);
      setUser(response.data.user);
    } finally {
      setIsLoading(false);
    }
  };

  const contextValue: AuthContextType = {
    user,
    isAuthenticated: !!user && !!getAccessToken(),
    isLoading,
    login,
    logout,
    refreshTokens,
  };

  return React.createElement(
    AuthContext.Provider, 
    { value: contextValue }, 
    children
  );
};
\end{lstlisting}

Almacenamiento asimétrico de tokens:

\begin{lstlisting}[language=typescript]
// Access token: localStorage (necesario para headers de API)
export const getAccessToken = (): string | null => {
  return localStorage.getItem('accessToken');
};

// Refresh token: httpOnly cookie (gestionada por navegador)
// No es accesible desde JavaScript, previniendo XSS
\end{lstlisting}

Justificación: La separación de almacenamiento aprovecha los mecanismos nativos de seguridad del navegador. El localStorage permite acceso para configurar headers de Authorization, mientras que httpOnly cookies asegura que el refresh token no sea expuesto a ataques XSS.

\subsubsection*{PBI-8: Control de Acceso Basado en Roles (RBAC)}

La implementación de RBAC sigue estándares de seguridad definidos por NIST \cite{rbac_nist}. El middleware de validación de roles protege endpoints específicos:

\begin{lstlisting}[language=typescript]
export const roleMiddleware = (allowedRoles: UserRole[]) => {
  return (req: Request, res: Response, next: NextFunction) => {
    if (!req.user || !allowedRoles.includes(req.user.role)) {
      return res.status(403).json({ 
        error: 'Acceso denegado - permisos insuficientes' 
      });
    }
    next();
  };
};
\end{lstlisting}

Ejemplo de uso en rutas protegidas por rol:

\begin{lstlisting}[language=typescript]
// Solo ADMIN puede crear usuarios
router.post('/register', 
  authMiddleware,                    // Validar autenticación
  roleMiddleware(['ADMIN']),         // Validar rol
  asyncHandler((req, res) => authController.register(req, res))
);

// ADMIN y MANAGER pueden listar usuarios
router.get('/users',
  authMiddleware,
  roleMiddleware(['ADMIN', 'MANAGER']),
  asyncHandler((req, res) => authController.listUsers(req, res))
);
\end{lstlisting}

Justificación: El modelo de RBAC centraliza lógica de autorización, facilitando auditoría y modificación de políticas de acceso. La verificación tanto de presencia del usuario como de validez del rol proviene del JWT payload, asegurando inmutabilidad de claims.

\subsubsection*{PBI-9: Endpoint /auth/register - Creación de Usuarios}

El endpoint de registro implementa un flujo controlado donde únicamente administradores crean usuarios:

\begin{lstlisting}[language=typescript]
async register(
  data: RegisterRequest,
  adminId: string
): Promise<{ user: User; tempPassword: string }> {
  // Validación de entrada
  const validated = RegisterSchema.parse(data);
  
  // Verificación de existencia previa
  const existingUser = await prisma.user.findUnique({
    where: { email: validated.email },
  });
  
  if (existingUser) {
    throw new Error('El email ya está registrado');
  }

  // Generación de contraseña temporal (10 caracteres aleatorios)
  const tempPassword = this.generateTempPassword();
  const hashedPassword = await bcryptjs.hash(tempPassword, 10);

  // Creación de usuario
  const newUser = await prisma.user.create({
    data: {
      email: validated.email,
      name: validated.name,
      password: hashedPassword,
      role: validated.role as UserRole,
      sedeId: validated.sedeId,
    },
  });

  // Log de auditoría
  await this.logAdminAction(adminId, 'USER_CREATED', newUser.id);

  return {
    user: this.excludePassword(newUser),
    tempPassword,
  };
}
\end{lstlisting}

Esquemas de validación con Zod \cite{zod_docs}:

\begin{lstlisting}[language=typescript]
export const LoginSchema = z.object({
  email: z.string()
    .email('Email inválido')
    .transform(e => e.toLowerCase().trim()),
  password: z.string()
    .min(6, 'Contraseña mínimo 6 caracteres'),
});

export const RegisterSchema = z.object({
  email: z.string()
    .email('Email inválido')
    .transform(e => e.toLowerCase().trim()),
  name: z.string()
    .min(2, 'Nombre requerido')
    .max(100, 'Nombre demasiado largo'),
  role: z.enum(['ADMIN', 'MANAGER', 'EMPLOYEE'])
    .default('EMPLOYEE'),
  sedeId: z.string().uuid('ID de sede inválido').optional(),
});
\end{lstlisting}

Justificación: La generación de contraseña temporal por el sistema previene debilidad de contraseñas iniciales mientras mantiene control administrativo. Al obligar cambio de contraseña en primer acceso, se garantiza que solo el usuario conoce su credencial de acceso. Zod proporciona validación declarativa con type inference automático en TypeScript.

\subsubsection*{Componente ProtectedRoute - Rutas Protegidas (Frontend)}

La implementación de rutas protegidas en el frontend enforza políticas de autorización a nivel de UI, reduciendo exposición accidental de interfaces restringidas:

\begin{lstlisting}[language=typescript]
interface ProtectedRouteProps {
  children: React.ReactNode;
  requiredRoles?: UserRole[];
}

const ProtectedRoute: React.FC<ProtectedRouteProps> = ({ 
  children, 
  requiredRoles 
}) => {
  const { user, isAuthenticated } = useAuth();

  // Redirección a login si no hay sesión activa
  if (!isAuthenticated) {
    return <Navigate to="/login" replace />;
  }

  // Validación de rol (si se especifican roles requeridos)
  if (requiredRoles && !requiredRoles.includes(user?.role!)) {
    return (
      <div className="p-4 bg-red-50 border border-red-200">
        Acceso denegado - permisos insuficientes
      </div>
    );
  }

  return <>{children}</>;
};
\end{lstlisting}

Configuración de React Router \cite{react_docs}:

\begin{lstlisting}[language=typescript]
const AppRouter = () => {
  return (
    <BrowserRouter>
      <Routes>
        {/* Rutas públicas */}
        <Route path="/login" element={<Login />} />
        
        {/* Rutas protegidas - acceso general */}
        <Route 
          path="/dashboard" 
          element={
            <ProtectedRoute>
              <Dashboard />
            </ProtectedRoute>
          } 
        />
        
        {/* Rutas protegidas - solo ADMIN */}
        <Route 
          path="/admin" 
          element={
            <ProtectedRoute requiredRoles={['ADMIN']}>
              <AdminPanel />
            </ProtectedRoute>
          } 
        />
      </Routes>
    </BrowserRouter>
  );
};
\end{lstlisting}

Justificación: La verificación de autenticación en ProtectedRoute proporciona una capa de seguridad adicional a nivel de UI. Aunque la seguridad primaria debe residir en el backend (validación de JWT), esta capa previene exposición accidental de interfaces y mejora experiencia de usuario al redirigir usuarios no autenticados.

\subsubsection*{Resumen Técnico Sprint 1}

El Sprint 1 estableció una base arquitectónica sólida basada en prácticas de seguridad reconocidas. La tabla \ref{tab:sprint1-tech-summary} resume tecnologías y patrones implementados:

\begin{table}[h]
    \centering
    \caption{Sprint 1 - Resumen de Arquitectura Técnica}
    \label{tab:sprint1-tech-summary}
    \vspace{0.3cm}
    \begin{tabularx}{\textwidth}{l X}
        \toprule
        \textbf{Componente} & \textbf{Especificación} \\
        \midrule
        Stack Backend & Node.js 20 + Express.js \cite{express_docs} + TypeScript \cite{typescript_docs} \\
        Base de Datos & PostgreSQL 16 \cite{postgresql_docs} + Prisma ORM \cite{prisma_docs} \\
        Autenticación & JWT (RFC 7519) \cite{rfc7519} con access/refresh tokens \\
        Hash de Contraseña & bcryptjs \cite{bcryptjs_docs} con cost factor 10 \\
        Validación de Entrada & Zod \cite{zod_docs} con type inference TypeScript \\
        Stack Frontend & React 18 \cite{react_docs} + Vite + Tailwind CSS \\
        Gestión de Rutas & React Router con protección por rol \\
        Seguridad CORS & Configuración con credenciales habilitadas \cite{cors_mdn} \\
        RBAC & Implementación según estándares NIST \cite{rbac_nist} \\
        DevOps & Docker \cite{docker_docs} + GitHub Actions CI/CD \\
        Control de Versiones & Git + GitHub (TFG-Multihub) \\
        Linting/Formatting & ESLint (airbnb-typescript) + Prettier \\
        \bottomrule
    \end{tabularx}
\end{table}

\subsubsection*{Resultados y Validación Sprint 1}

La completitud del Sprint 1 fue validada mediante:

\begin{itemize}
    \item \textbf{Pruebas Funcionales}: Endpoint \texttt{/api/auth/login} responde correctamente con JWT tokens válidos y user data
    \item \textbf{Verificación de Seguridad}: Validación de httpOnly cookies para refresh tokens y CORS configuration correcta
    \item \textbf{Cobertura de Tipado}: TypeScript strict mode sin errores de compilación en frontend y backend
    \item \textbf{Infraestructura}: Docker Compose services (PostgreSQL, Backend, Frontend) todos operacionales
    \item \textbf{Datos de Prueba}: Database seeded con usuarios de prueba en todos los roles (ADMIN, MANAGER, EMPLOYEE)
    \item \textbf{Endpoints API}: 7 endpoints de autenticación completamente funcionales
\end{itemize}

\textbf{Flujo de autenticación end-to-end validado:}
\begin{enumerate}
    \item Usuario navega a \texttt{http://localhost:3000/login}
    \item Ingresa credenciales que son validadas por esquema Zod contra restricciones definidas
    \item Submit realiza POST request hacia \texttt{/api/auth/login} con credenciales JSON
    \item Backend localiza usuario por email en PostgreSQL y valida contraseña contra hash bcryptjs
    \item JWT tokens generados con claims (userId, email, role) y expiración configurada
    \item Access token retornado en respuesta JSON; refresh token en httpOnly cookie
    \item Frontend almacena accessToken en localStorage para futuras requests API
    \item User automáticamente redirigido a dashboard con sesión activa y user info en contexto
    \item Logs de auditoría registran acceso exitoso (timestamp, IP, user agent)
    \item Logout limpia localStorage y elimina refresh cookie vía Set-Cookie header
    \item Subsequent requests usan accessToken en Authorization header (Bearer scheme)
\end{enumerate}

\subsubsection*{Decisiones de Arquitectura Fundamentales}

Las decisiones técnicas en Sprint 1 fueron justificadas por:

\begin{itemize}
    \item \textbf{JWT sobre Session Storage}: Stateless authentication facilita escalabilidad horizontal en microservicios futuros \cite{rfc7519}
    \item \textbf{Dual Token Strategy}: Access tokens cortos (15m) reduce ventana de riesgo; refresh tokens largos (7d) mantienen UX
    \item \textbf{httpOnly Cookies}: Previene XSS attacks eliminando acceso JavaScript a refresh tokens críticos
    \item \textbf{Contrast-time Password Comparison}: Uso de bcryptjs.compare() previene timing attacks en validación de credenciales
    \item \textbf{Role-Based Access Control}: Implementación granular permite asignación flexible de permisos con política centralizada
    \item \textbf{TypeScript Strict Mode}: Elimina categorías enteras de bugs comunes (null/undefined, implicit any types)
    \item \textbf{Prisma ORM}: Schema como source of truth; migraciones versionadas permiten auditoría histórica de cambios DB
\end{itemize}

\subsection*{Sprint 2 (Feb 23-Mar 8): RBAC y Gestión de Usuarios}
\textbf{PBIs:} 8-14 (30 SP)

\subsection*{Sprint 3 (Mar 9-22): Gestión de Turnos}
\textbf{PBIs:} 20-24 (28 SP)

\subsection*{Sprint 4 (Mar 23-Apr 5): Dashboard CEO}
\textbf{PBIs:} 15-19 (26 SP)

\subsection*{Sprint 5 (Apr 6-19): Permutas + Inventario}
\textbf{PBIs:} 25-33 (30 SP)

\subsection*{Sprint 6 (Apr 20-May 4): Nóminas, Feedback y Testing}
\textbf{PBIs:} 34-42 (30 SP)



% ==========================================================
% BIBLIOGRAFÍA
% ==========================================================
\printbibliography[heading=bibintoc, title={Bibliografía}]

\end{document}